\section{Resumen de la información}
% Ideal: Entre 500 y 600 palabras. Total:160

\subsection{Los treintas, política y cine}
\begin{compactitem}
     \item 
     Desde inicios de la década, ``los grupos de poder ya tenían [\ldots] una clara 
     consciencia de la capacidad del cinematógrafo como medio de comunicación de masas y
     como herramienta de construcción de imaginarios sociales, políticos y culturales"\autocite[108]{piedras_narrativas_2012}
    \item
    Su función se torna didáctica\autocite[439]{monsivais_notas_????}, ``ya que ya no será informar a la población sobre los hechos de la
    Revolución, sino figurar y construir narrativas sobre esta, con el objetivo [\ldots], de asentar ciertos imaginarios e interpretaciones sobre la historia reciente''\autocite[108]{piedras_narrativas_2012}
   \item
   Sin embargo, no podemos interpretar una obra ``directamente como el
   [\ldots] punto de vista de sus responsables, ni tampoco, como agentes culturales que
   transmiten sin más el sistema de valores y las concepciones políticas de un entorno social'', sino que tambíen ``expresan 
   las tensiones sociales,políticas y culturales y los imaginarios de la sociedad que las produce"\autocite[109]{piedras_narrativas_2012}
    \item
    Desde este período ``la Revolución, [\ldots] se convertirá en una narrativa recurrente del cine [\ldots],adquirirá múltiples valoraciones e interpretaciones de acuerdo con el
    contexto sociopolítico en que se realizan las obras y el posicionamiento ideológico de sus
    autores".\autocite[110]{piedras_narrativas_2012}
    \item
    Durante el periodo Cardenista se ``impulsó una serie de políticas [\ldots] que no siempre resultaron
    bienvenidas por los sectores medios y acomodados de la sociedad". \autocite[114]{piedras_narrativas_2012}
    \item
    Se considera como un período en el cual las disputas de clase y los conflictos sociales movilizados por la
Revolución no se habían terminado de dirimir\autocite[112]{piedras_narrativas_2012}
    \item ``La versión oficial y pública [de la historia] termina 
siendo la del cine"\autocite[440]{monsivais_notas_????}
\end{compactitem} 




\subsection{Los cuarentas, política y cine}
\begin{compactitem}
     \item 
     A inicios de los años 40, aumentó considerablemente la popularidad del cine mexicano, así como su producción.\autocite[522]{peter_desarrollo_2008}, 
     \item Por esto, el cine se torna más comercial ``desde finales de los treinta  el comercialismo había desplazado del cine al nacionalismo revolucionario''\autocite[443]{monsivais_notas_????}
     \item
     ``México (declara) la guerra al Eje y el presidente Ávila Camacho (pide) la absoluta armonía entre todas las clases 
     sociales unidad nacional para enfrentar al enemigo externo.''\autocite[116]{piedras_narrativas_2012}
     \item 
     1943 llamado el ``gran año'' del cine Mexicano, por el incremento en la producción y la creación del Banco Cinematográfico. %citation deb rev2

\end{compactitem} 

\subsection{Fernando de Fuentes}
    \begin{compactitem}
        \item Durante la década de los treinta filma y estrena su ``Trilogía
        de la revolución", la cual, tiene un éxito limitado.\autocite{piedras_narrativas_2012}
        \item Sus películas tienden a abordar una perspectiva crítica sobre la revolución,aunque limitada, puesto que no discute sus motivaciones. \autocite[113]{piedras_narrativas_2012}
        \item En el año de 1936 se estrena su película ``Vamonos
        con Pancho Villa"\autocite{fernandez_vamonos_1936}.autocite{_vamonos_????}
    \end{compactitem} 

\subection{Emilio ``El indio" Fernández}
    \begin{compactitem}
    \item Sus pelí--- culas fueron internacionalmente las más importantes durante la década de los cuarenta. Mediante ellas, se convirtió Pedro Armendáriz en un ícono nacional, quien  representaba el ideal del hombre virtuoso, de confianza y sin pretensiones.\autocite[522]{peter_desarrollo_2008}
    \item Considerado como uno de los directores más representativos del Cine de Oro Mexicano a nivel mundial, ganador de múltiples galardones nacionales, y otros  internacionales. \autocite[9]{aguilar_construccion_2014}
    \item ``Se ve influenciado por el movimiento vasconcelista, el cual pugnaba por la exaltación de los valores patrióticos y el fomento a la educación. Por esta razón, su cine exalta el nacionalismo, e incorpora la imagen nacional mediante estereotipos, sin embargo, su cine puede llegar a ser también promotor del indigenismo y de la denuncia social.''\autocite[9-10]{aguilar_construccion_2014}.
    \item Se considera que ``en sus películas la Revolución tiene una centralidad menor y se convierte en el
    trasfondo espacial y  de la trama''\autocite[111]{piedras_narrativas_2012}, además tienen un tono más conciliador que crítico\autocite{piedras_narrativas_2012}
    \item Durante el año de 1943 estrena la película de ``Flor
    Silvestre"\autocite{fernandez_flor_1946}.\autocite{_flor_????}
    \end{compactitem} 

(572 palabras)
\pagebreak

