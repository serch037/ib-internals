\section{Resumen de la información}
% Ideal: Entre 500 y 600 palabras. Total:160
%TODO: Dividir en subsecciones, resumir
\subsection{Contexto Sociopolítico}
\begin{compactitem}
\item 
\hlc[LimeGreen]{
El Banco Cinematográfico patrocinó obras nacionalistas como las del Indio Fernández
}
\autocite[10]{garcia_riera_historia_1992}. 
%\item ``[\ldots] los productores alentados por el Banco Cinematográfico se probaron alentados por un cine nacionalista [\ldots] como la Films Mundiales [\ldots] [cuyo líder tenía un]  especial interés en promover un cine como [\ldots] [el de] Emilio Fernández [\ldots]''\autocite[10]{garcia_riera_historia_1992}
%\item ``A partir de 1947, primer año del sexenio del presidente Migel Alemán, un recurso de estandarización hizo de las películas mexicanas, es su gran mayoría, eslabones de una cadena.''\autocite[105]{garcia_riera_historia_1993}
\item  
\hlc[Apricot]{
Los ideales de Ávila Camacho incluían la unidad nacional y la 
conciliación
}
\autocite[365-366]{sanchez_vi._2010}.
%\item ``[\ldots] el presidente Ávila Camacho tenía que ser conciliador [\ldots]''\autocite[365]{sanchez_vi._2010}
%\item El comercialismo desplazó al cine revolucionario hacia finales de los treintas\autocite[439]{cosio_villegas_notas_1976}.
%\item ``A fines de los treintas, el comercialismo ya ha desplazado del cine al precario nacionalismo revoluionario''\autocite[439]{cosio_villegas_notas_1976}
\item 
\hlc[Bittersweet]{
En este periodo se sustituye a la síntesis histórica por la fragmentación
}
\autocite[440]{cosio_villegas_notas_1976}.
%\item ``[\ldots] [las películas como Flor Silvestre suelen] gloriarse en la fragmentción porque ha desistido de la síntesis''\autocite[440]{cosio_villegas_notas_1976}
%\item ``Con Aleman 8.3'' se dedicó a la educación pública. \autocite[451]{pablos_escuela_1998-1}
%\item Las reformas revolucionarias se abandonaron progresivamente. \autocite[62]{consuelo_rangel_ley_2006}
%\item ``[\ldots] muchas de las reformas [\ldots] se fueron abandonando poco a poco  bajo [\ldots] Ávila Camacho''\autocite[62]{consuelo_rangel_ley_2006}
\item 
\hlc[magenta]{
El BC y el jefe de censura prohiben el cine crítico-histórico.
}
\autocite[179]{guerrero_imagen_2005}
%\item ``El BC\footnote{Banco Cinematográfico} y el jefe de censura del cine nacional de la época [\ldots] prohiben [\ldots] hacer películas crítico-históricas.''\autocite[179]{guerrero_imagen_2005}
\item 
\hlc[red]{
La administración Avila-Camachista invirtió en la creación de una cultura nacional.
}
\autocite[35]{tierney_myths_2002}
\item \hlc[AliceBlue]{
El cine era fuertemente católico pero permitía transgresiones sutiles}
\autocite[48]{tierney_myths_2002}
%\item ``(Felipe Gregorio del Castillo, jefe del departamento de censura) No es posible permitir la exhibición de películas que denigren a México [\ldots] [tampoco] en las que sólo se pintan  las carácterísticas de violencia y de vicio [\ldots] del pueblo mexicano [\ldots] tampoco es posible que se sigan presentando los aspectos vergonzosos de la revolución [\ldots]''\autocite[111]{garcia_riera_historia_1992}
\item 
\hlc[cyan]{
El clero influyó notoriamente en el cine.
}
\autocite[12]{garcia_riera_historia_1992}
%\item ``A lo largo del sexenio de Ávila Camacho resulto notoria la fuerte influencia del clero en la industria cinematográfica''\autocite[12]{garcia_riera_historia_1992}
\item 
\hlc[Mahogany]{
El discurso sociopolítico de los años cuarenta ignora la violencia de la Revolución y exalta su rol como forjador de paz
}
\autocite[370]{sanchez_vi._2010}.
%\item ``[\ldots] la autenticidad que elogiaba la crítica de los años cuarenta se entiende como un discurso que pasa por alto la violencia y resignifica la Revolución como el proceso forjador de la paz nacional''\autocite[370]{sanchez_vi._2010}
\item 
\hlc[blue]{
El conservadurismo es reflejado en el cine.
}
\autocite[179]{guerrero_imagen_2005}
%\item ``[\ldots] el [\ldots] conservadurismo [\ldots] se refleja en el cine [\ldots]''\autocite[179]{guerrero_imagen_2005}
\item 
\hlc[green]{
Las películas de esta época instalan en el imaginario social la ideología de la mexicanidad.
}
\autocite[12]{silva_escobar_epoca_2011}
%\item ``Las películas de la Época de Oro naturalizan [\ldots] la ``mexicanidad'' [\ldots] instalan en el imaginario social [\ldots] [esa] ideología''\autocite[12]{silva_escobar_epoca_2011}
\item 
\hlc[yellow]{
1943 es ``el gran año del cine mexicano'', pues se coloca como ``una verdadera industria''.
}
\autocite[7]{garcia_riera_historia_1992}
\end{compactitem} 


\subsection{Emilio ``El indio" Fernández}
    \begin{compactitem}
    \item 
    \hlc[BlueViolet]{
    El cine hará más bien a México que un fusil
    }
    \autocite[16:30]{soler_serrano_emilio_1976}.
    \item 
    \hlc[OrangeRed]{``[\ldots] el Indio Fernández ha sido un mitificador y un mitómano''
    }
    \autocite[445]{cosio_villegas_notas_1976}
    \item
    \hlc[DarkOrchid]{
     El cine del Indio sintetiza el nacionalismo cultural
     }
     \autocite[444]{cosio_villegas_notas_1976}
    %\item ``El clímax del nacionalismo cultural cinematográfico es la obra del Indio Fernández''\autocite[444]{cosio_villegas_notas_1976}
    %\item Funda una ``escuela mexicana de cine''\footnote{En sentido figurado, refiriéndose al estilo y su influencia sobre el resto de los directores}\autocite[133]{mora_mexican_1978-2}.
    %\item Fundada una ``escuela mexicana de cine''\footnote{En sentido figurado, refiriéndose al estilo} y su periodo de fama abarcó la segunda mitad de los cuarentas hasta los cincuentas\autocite[133]{mora_mexican_1978-2}.
    %\item{``[\ldots] se daba cuenta de que los responsables de la educación nacional no seguían su propuesta [la del proyecto revolucionario], pensaba que equivocaban el rumbo}''\autocite[466]{pablos_escuela_1998-1}
    \item 
    \hlc[Melon]{
    ``[\ldots] quiere realizar un cine crítico''
    }
    \autocite[466]{pablos_escuela_1998-1}
    \item 
    \hlc[Mulberry]{
    ``[\ldots] tiene una visión propia del futuro de México''
    }
    \autocite[158]{aguilar_construccion_2014-1}
    \item 
    \hlc[CadetBlue]{
    Su nacionalismo es multifacético y exaltado en sus películas
    }
    \autocite[127]{aguilar_construccion_2014-1}.
    %\item ``El nacionalismo de ``El Indio'' tiene varios rostros que son exaltados en sus películas [\ldots]''\autocite[127]{aguilar_construccion_2014-1}
    \item 
    \hlc[Cerulean]{
    Sus películas incorporan mensajes socialmente importantes
    }
    \autocite[72]{tierney_myths_2002}
    %\item Participó directamente en la lucha revolucionaria  %TODO: fix reference
    \end{compactitem} 

\subsection{Flor Silvestre}
\subsubsection{Repartición}
\subsubsection{Métodos}
\subsubsection{Relación con la iglesia}
    \begin{compactitem}
    %\item ``\emph{Flor Silvestre} facilitó [\ldots] la formación del primer equipo de lujo del cine mexicano.''\autocite[19]{garcia_riera_historia_1992}
    %\item ``Estrenada [\ldots] [en] 1943''\autocite[17]{garcia_riera_historia_1992}
    %\item ``Producción (1943): Films Mundiales [\ldots] Fotografía: Gabriel Figueroa [\ldots] Interprétes: Dolores del Río [\ldots] Perdro Armendáriz''\autocite[16]{garcia_riera_historia_1992}
    %\item Ofrece una versión revisionista de la Revolución.\autocite[369]{sanchez_vi._2010}
    %\item ``\emph{Flor Silvestre} y \emph{Enamorada} [\ldots] [ofrecen] versiones de la Revolución revisionistas [\ldots]''\autocite[369]{sanchez_vi._2010}
    \item 
    \hlc[Aquamarine]{`` [\ldots] la película [\ldots] pasa por alto [\ldots] la guerra entre el Estado y la Iglesia [\ldots] ''}
    \autocite[370]{sanchez_vi._2010}.
    \item 
    \hlc[Rhodamine]{
    El tema de la película (y \emph{Enamorada}) es la armonía entre las clases sociales. Y refleja su contexto cultural más que los detalles de la revolución}
    \autocite[365]{sanchez_vi._2010}.
    %\item ``Es precisamente el la ``absoluta armonía entre todas las clases sociales'' el tema de las dos películas [Flor Silvestre y Enamorada] [\ldots], que tienen que ver más con el \emph{Zeitgeist} de fondo que con la Revolución, que supuestamente constituye el telón de fondo.''\autocite[365]{sanchez_vi._2010}
    \item \hlc[CarnationPink]{
    ``La tierra es de quién la trabaja y sueña y sufre en ella''
    }
    \autocite[11:14]{fernandez_flor_1943}
    \item 
    \hlc[RoyalPurple]{
    Periódicos sobre la revolución presentan la lucha con una marcha\footnote{Específicamente, la \emph{Marcha Dragona}} de honor mexicana como fondo musical. Estos tienen los encabezados:``La revolución lucha por justicia social''; ``La revolución cunde por toda la república''; ``Madero pide unificación''
    }
    \autocite[42:38-43:20]{fernandez_flor_1943}.
    \item \hlc[Turquoise]{
    El siguiente paso de la revolución es detener a los bandidos desatados durante el proceso de la lucha, quienes pretenden ser revolucionarios
    }
    \autocite[48:07-48:40]{fernandez_flor_1943}.
    %\item ``hay que acabar con el bandidaje, los jefes quieren controlar lo que desatamos [\ldots] limpiar a esta tierra de los bandidos, que a la sombra de nuestra causa roban y asesinan, esos son los peores enemigos de la revolución''\autocite[48:07-48:40]{fernandez_flor_1943}
    %\item ``A la sombra de la revolución que lucha por ideales de mejoramiento de los humildes muchos bandidos azotan el país''\autocite[1:21:40]{fernandez_flor_1943}
    %\item ``Es considerado el melodrama revolucionario por excelencia''\autocite[178]{guerrero_imagen_2005}
    %\item ``se proyecta [\ldots] a la revolución como una gran amenaza para la familia, la tradición y los valores patriarcales''\autocite[178]{guerrero_imagen_2005}
    \item 
    \hlc[SpringGreen]{
    Presenta ambigüedades respecto a los revolucionarios, pero es nacionalista y les ofrece simpatía
    }
    \autocite[87]{demello_unfinished_2001}
    %\item Fue la primera película del equipo que formó y fijó el modelo para las obras subsecuentes\autocite[86]{demello_unfinished_2001}
    %\item Fue el primer éxito del director\autocite[19]{costa_cine_2010}
    %\item ``En principio [\ldots] la obra [\ldots] debía ser eminentemente didáctica y escolarmente aleccionadora''\autocite[28]{blanco_aventura_1993}
    \item 
    \hlc[TealBlue]{
    ``[\ldots] [en el film] la lucha armada tuvo un buen principio [\ldots] [pero] provocaba desórdenes y víctimas incocentes''
    }
    \autocite[36]{blanco_aventura_1993}.
    %\item El propósito era presentar una versión de México y la revolución mexicana\autocite[23:19]{soler_serrano_emilio_1976}
    \item
    \hlc[RedOrange]{
    Sobre la sangre de quienes creyeron en el bien y la justicia se levanta el México de hoy
    }
    \autocite[1:30:10]{fernandez_flor_1943}.
    \end{compactitem} 

\subsection{Enamorada}
\subsubsection{Repartición}
\subsubsection{Métodos}
\subsubsection{Relación con la iglesia}
    \begin{compactitem}
    \item \hlc[MidnightBlue]{
    ``En \emph{Enamorada}, Iglesia y Revolución se reconilcan''
    }\autocite[377]{sanchez_vi._2010}.
    \item 
    \hlc[Gray]{
    ``Yo no les llamaría bandidos, son revolucionarios''
    %, otro le contesta: ``Están destrozando al país y al comercio''
    }
    \autocite[5:31-5:46]{fernandez_enamorada_1946}.
    \item \hlc[Dandelion]{
    ``[\ldots] esos son los verdaderos traidores, las sangijuelas que se alimentan chupando la sangre de sus hermanos''
    }
    \autocite[14:11]{fernandez_enamorada_1946}
    \item \hlc[Periwinkle]{
    ``Es un hombre extraño, no sé si es malo o es bueno, pero se le ve en los ojos que cumple su palabra''
    }
    \autocite[39:00]{fernandez_enamorada_1946}
    \item \hlc[YellowOrange]{
    Los revolucionarios izan la bandera nacional y marchan frente a un monumento que conmemora la independencia
    }
    \autocite[03:04]{fernandez_enamorada_1946}.
    %\item Gobernador recibe a revolucionarios y les alaba llamándoles libertadores.\autocite[3:52]{fernandez_enamorada_1946}
    \item \hlc[Orchid]{
    Protagonista indaga sobre los ricos del pueblo y les exige sus bienes amenazándolos, mientras tanto, una niña vestida de adelita se sienta a su lado
    }
    \autocite[4:57-5:13]{fernandez_enamorada_1946}.
    \begin{compactitem}
    \item \hlc[SkyBlue]{
    Dice que no le interesa para él, sino para la causa
    }
    \autocite[21:38]{fernandez_enamorada_1946}.
    \end{compactitem}
    \item \hlc[Orange]{
    El cura relata que él y el general revolucionario se hicieron amigos en un seminario religioso
    }\autocite[4:57-5:13]{fernandez_enamorada_1946}.
    %\item Todas las iglesias cerradas.\autocite[8:50]{fernandez_enamorada_1946}
    %\item El cura dice acompañar a los ricos y sermonea al general sobre su trato.\autocite[9:30]{fernandez_enamorada_1946}
    %\item ``Van a tener que confesar con su nuevo padre''\autocite[9:56]{fernandez_enamorada_1946}
    %\item Un comerciante le dice al protagonista que no entiende la revolución y solo quiere seguir trabajando, este le contesta que aun sin entenderla, se aprovecha de ella subiendo los precios. Le explica que el objetivo de la revolción es liberar a los hombres y no enriquecer a unos pocos. Lo fusila cuando el comerciante le ofrece a su esposa a cambio de su vida.\autocite[13:01-15:00]{fernandez_enamorada_1946}
    %\item Norteamericano dice: ``Yo sí creo en la revolución, déjeme ayudarlos a ustedes y ustedes me ayudan a mí''\autocite[28:46]{fernandez_enamorada_1946}
    \item \hlc[SeaGreen]{
    Protagonista le ofrece al maestro (a quien el gobierno no había pagado) una indemnización y un mejor salario.
    }
    \autocite[18:30]{fernandez_enamorada_1946}
    %\item Protagonista alaba a Juárez e imparte un discurso en el cuál iguala la lucha de la iglesia por salvar a los hombres con el de los revolucionarios. Desea crear el cielo en la tierra. ``La patria es el cielo de la tierra'' y el cielo ``la patria eterna''. Insinúa que la iglesia ha perdido su camino y debe actuar como los antiguos misioneros.\autocite[22:31-25:17]{fernandez_enamorada_1946}
    %\item Protagonista interpreta un cuadro de los reyes magos como una alegoría de la igualdad de clases a través del amor.\autocite[1:07:19]{fernandez_enamorada_1946}  
    \item \hlc[Thistle]{
    Revolucionarios se retiran con una marcha patriótica sin pelear y devuelven algunas de las riquezas tomadas, pues es más digno
    }
    \autocite[1:27:21-1:35:20]{fernandez_enamorada_1946}.
    \end{compactitem} 

% \subsection{Río escondido}
%    \begin{compactitem}
%    \item ``Producción(1947): Raúl de Anda [\ldots] Fotografía: Gabriel Figueroa. [\ldots] Intérpretes: María Félix [\ldots] Fernando Fernándes [\ldots]''\autocite[143]{garcia_riera_historia_1993}
%     \item ``Estrenada [\ldots] [en] 1948''\autocite[144]{garcia_riera_historia_1993}
%     \item ``Solo una película de 1947 ganó premios internacionales [\ldots] \emph{Río Escondido}''\autocite[108-109]{garcia_riera_historia_1993}
%     \item `` [\ldots] la película había sido objetada [\ldots] por un cronista más patriotero y conservador [\ldots] también la criticó el izquierdista [\ldots]''\autocite[144]{garcia_riera_historia_1993}%
%     \item \emph{Río escondido} es la película que mejor revela los ideales de emilio Fernández\autocite[171]{mora_mexican_1978-2}.
%     \item Recibió tanto crítica como elogios\autocite[170-171]{mora_mexican_1978-2}.
%     \item La protagonista es asignada su tarea por el propio presidente Miguel Alemán \autocite[168]{mora_mexican_1978-2}.
%     \item Solo es visible la nuca y la sombra del perfil del presidente \autocite[12:15-12:40]{fernandez_rio_1947}
%     \item Epígrafe:``Esta historia no se refiere precisamente al México de hoy ni ha sido nuestra intención situarla dentro de él.''\autocite[00:00]{fernandez_rio_1947}
%     \item ``Hagamos juntos un esfuerzo para salvar a nuestro pueblo''\autocite[12:51]{fernandez_rio_1947}
%     \item ``Mientras [\ldots] no salgan [\ldots ]del analfabetismo, no podremos levantarnos''\autocite[12:27]{fernandez_rio_1947}
%     \item ``[\ldots] crea mitos [\ldots] del triunfo de la Revolución y así infunde un sentimiento nacionalista''\autocite[66]{consuelo_rangel_ley_2006}
%     \item ``[\ldots] Monsiváis ha reducido a  \emph{Río escondido} a la categoría de un cuento de hadas [\ldots]''\autocite[78]{blanco_aventura_1993}
%     \end{compactitem} 

%(717 palabras; debo llegar a 666 o 715)
\pagebreak

