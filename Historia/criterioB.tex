\section{Resumen de la información}
% Ideal: Entre 500 y 600 palabras. Total:160
\subsection{Contexto histórico}
\begin{compactitem}
    \item Durante los 50s, ``el país experimentó una acelerada modernización que fue celebrada como el milagro mexicano”\autocite{SoledadLoaeza2010}
    \item ``Aparentaba dirigirse hacia uno de carácter más internacionalmente relevante"\autocite[1]{Kehoe2013}
    \item El gobierno implementa políticas centralistas; con el propósito de promover la industria y acelerar el desarrollo del país ``se hizo presente el estado en todo el territorio nacional mediante la extensión y aplicación de leyes y reglamentos'' \autocite{SoledadLoaeza2010}
    \item Para finales de los cuarenta, era evidente que la revolución, con su  promesa de justicia social, había fracasado.\autocite[1]{jones_bunuel_2006} 
\end{compactitem} 
\subsection{El cine durante los 40s}
\begin{compactitem}
     \item Aumentó considerablemente la popularidad del cine mexicano, así como su producción, a inicios de los años 40.\autocite[522]{peter_desarrollo_2008}

\end{compactitem} 
\subsection{El cine durante los  50s}
\begin{compactitem}
    \item El cine de la época ``centró su atención en un público
    con distinto perfil: la clase media''\autocite[96]{BarcenasSanchez2014}
    \item La cinematografía nacional cada vez estaba más por los suelos, por la competencia extranjera se veían obligados a producir hasta tres películas por semana, que eran como resultado un ``churro'' era un cine rutinario y vulgar, carente de imaginación \autocite[12]{MillanHernandez2004}.
    \item La representación del país en el cine no correspondía con la realidad social, apoyado por el estado, glorificaba el campo, los valores conservadores y  condenaba a la ciudad.\autocite[1]{jones_bunuel_2006} 
    \item La tendencia fue de abandonar los temas y estilos experimentales mientras se explotaban las temáticas comerciales y rentables \autocite[24]{baugh_developing_2004}
    \item El cine cae en fórmulas repetidas y productos de baja calidad.\autocite[522]{peter_desarrollo_2008} 
\end{compactitem}

    \subsubsection{Censura en el cine durante la época}
    \begin{compactitem}
    \item Creció la censura de temas políticos y económicos tratados desde perspectivas críticas o donde fuera evidente que el hombre promedio tenía pocas probabilidades de ascenso social \autocite[100]{BarcenasSanchez2014}
    \item Se redujo la libertad de prodcción mediante leyes proteccionistas, cuotas de exhibición y el monopolio de los estudios.\autocite[25]{baugh_developing_2004}
    \item Los proyectos provocativos eran rechazados por los inversionistas debido a sus intereses económicos\autocite[28]{baugh_developing_2004}
    \item El estado intervenía en la producción y distribución mediante el Banco Nacional Cinematográfico, y la Dirección General de Cinematografía\autocite[29]{baugh_developing_2004}. De esta manera, podía eliminar de las películas contenido que considerase inapropiado, como "rebeldías políticas, y diálogos insinuantes"\autocite[28]{baugh_developing_2004}
    \end{compactitem} 

\subection{Emilio "El indio" Fernández}
    \begin{compactitem}
    \item Sus películas fueron internacionalmente las más importantes durante la década de los cuarenta. Mediante ellas, se convirtió Pedro Armendáriz en un ícono nacional, quien  representaba el ideal del hombre virtuoso, de confianza y sin pretensiones.\autocite[522]{peter_desarrollo_2008}
    \item Considerado como uno de los directores más representativos del Cine de Oro Mexicano a nivel mundial, ganador de múltiples galardones nacionales, y otros  internacionales. \autocite[9]{aguilar_construccion_2014}
    \item Se ve influenciado el movimiento vasconcelista, el cual pugnaba por la exaltación de los valores patrióticos y el fomento a la educación. Por esta razón, su cine exalta el nacionalismo, e incorpora la imagen nacional mediante estereotipos, sin embargo, su cine puede llegar a ser también promotor del indigenismo y de la denuncia social.  \autocite[9-10]{aguilar_construccion_2014}.
    \item La cinta ``La cucharacha'' se asocia con él por por las convergencias en cuanto a temática, ideología y técnica cinematográfica \autocite[20]{aguilar_construccion_2014}.
    \end{compactitem} 
(488 palabras)
\subsection{Comentarios}
Falta revisar Monsiváis, fuentes anteriores, y fuentes especializadas sobre la representación del periodo revolucionario, notablemente "La luz y la guerra"
\pagebreak
