\section{Conclusión}
Como hemos podido observar, la obra del Indio del periodo seleccionado expresa un explícito sentimiento nacionalista. Y se alinea casi siempre al discurso nacionalista oficial, a pesar de lo cual se permite señalar superficialmente algunas flaquezas de la revolución, tales como sus ocacionales excesos de violencia, o el haber sembrado --aunque indirectamente-- el caos. Asimismo, ignora episodios  de la revolución, como la guerra cristera,  que contradicen la visión histórica oficial. Al mismo tiempo, los objetivos de la revolución son presentados como exclusivamente sociales y filantrópicos, ya que pretendían reunir a todas las clases sociales así como repartir equitativa y justamente la tierra y la riqueza.

Por lo tanto, podemos concluir que en las películas seleccionadas se mitifican  las causas y métodos de la revolución para producir un sentimiento de orgullo nacionalista en el espectador mexicano, y exportar una visión de México que presentara al gobierno posrevolucionario como uno fundamentado en la justicia social.

%Por lo tanto concluimos que en la obra cinematográfica del director, se favorece al discurso nacionalista oficial, con las limitaciones de  muy ocasionales y sutiles, pero específicas críticas a la administración política de su contexto histórico. Esto ocurre primordialmente mediante una representación del pasado revolucionario de México que omite ciertos detalles históricos, como la guerra Cristera, o reescribe otros, como los conflictos internos entre revolucionarios.

(Conteo de palabras 153)
(Conteo total 1988)

