\section{Conclusión}
Como hemos podido observar, la obra del Indio expresa un explícito sentimiento nacionalista. A pesar de que en ocasiones su visión de la revolución se alínea, al discurso oficial, identificamos también momentos en las cuáles la visión del director, aunque continúa siendo nacionalista, diverge políticamente de la versión oficial. De esta manera, el director es capaz de crear una visión propia. Sin embargo, estos son solo breves momentos en los cuáles el autor se permite hacer una muy sutil réplica al gobierno de su contexto.

Por lo tanto concluimos que en la cinematografía del director, se favorece al discurso nacionalista oficial, con las limitaciones de  muy ocasionales y sutiles, pero específicas  críticas a la administración política de su contexto histórico. Esto ocurre primordialmente mediante la representación del pasado revolucionario de México y el presente posrevolucionario.

%(Conteo de palabras 133)
%(Conteo total 2481)
