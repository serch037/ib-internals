\section{Conclusión}
Como hemos podido observar, ambas películas presentan aspectos similares de la revolución, abordan en mayor o menor medida problemáticas como la violencia, las motivaciones de los revolucionarios, y los resultados del conflicto. Sin embargo, esta elección temática denota el límite de similitues entre ambas, ya que difieren considerablemente en cuanto al tratamiento que les dan, y por tanto en el mensaje que transmiten.

El discurso narrativo revolucionario también ilustró algunos aspectos de la sociedad en su momento de estreno, por lo que la comparacion y el contraste de las obras nos es útil para oobservar el desarrollo de la sociedad méxicana en el periodo de estudio desde la óptica cultural. Por ejemplo, nos percatamos del progresivo cierre en cuanto a la riqueza del comentario crítico y su compromiso por versiones oficiales y fórmulas melodramáticas propias de la propia industrialización del país. 
 
Concluímos que el estudio de la producción cinematográfica es una importante herramienta historiográfica puesto que nos permite ver en acción distintos aspectos de la sociedad que les dio origen. 

(Conteo de palabras 171)
