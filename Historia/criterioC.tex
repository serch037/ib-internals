\section{Evaluación de Fuentes}
% Ideal: Entre 200 y 400 palabras. Total:
\subsection{¡Vámonos con Pancho Villa!}
La primera película que discutiremos es \textit{Vamonos con Pancho Villa}\autocite{fernandez_vamonos_1936}, por el director mexicano Fernando De Fuentes. 
Es considerada como la primera ``superproducción'' Mexicana, ya que contó con un presupuesto de más de un millón de pesos.%citation needed
Asímismo, el gobierno de Lázaro Cárdenas apoyó la producción con el préstamo de recursos económicos%footnote about CLASA
y escenográficos.
Comenzó rodaje en Noviembre de 1935%citation needed 
, pero por motivos de salud del director y causas económicas, no pudo ser estrenada sino hasta Diciembre del próximo año%citation
; para entonces, su película \textit{Allá en el Rancho Grande} habría arrasado con las taquillas. %citation
En 1982 se descrubrieron escenas adicionales que componen un final alternativo. 

Es una fuente de inmenso valor por que proviene de una de las figuras más importantes del cine mexicano y fue filmada durante un periodo en el cuál la censura nacionalista aún no entraba en completo vigor y la producción cinematográfica no se desvirtuaba por razones económicas.%citations 
Por lo tanto, nos permite conocer una mirada crítica del movimiento revolucionario que proviene de una de las más importantes figuras culturales de su contexto. 

Sin embargo, en el momento de su estreno %footnote about popularity
la película no fue difundida satisfactoriamente y por ello no podemos conoocer la reacción popular hacia ella. Además, como se desconocen los motivos de la omisión de la última escena no podemos saber si fue una  decisión artística del director, o una muestra de censura gubernamental.




(236 palabras)
\subsection{Flor Silvestre}
La segunda obra a analizar es \textit{Flor Silvestre}, del director mexicano
Emilio ``el Indio'' Fernández. Fue estrenada en 1943. Cuenta con fotografía de
Gabriel Figueroa y las actuaciones de Pedro Armendáriz y Dolores del Río. 
Es considerada como la obra maestra de este equipo de procucción %citation Ayala


Es valiosa por ser considerada como la única producción melodramática que
entabla una reflexión sobre el movimiento revolucionario. %citation needed
Además, por su naturaleza como producción conjunta nos permite observar la
manera en la cuál la ideología del director se entrecruza con la de su contexto. 
Finalmente, fue estrenada en un momento clave para la historia del cine
% citation 
y por su equipo de producción (Figueroa-Félix-Arméndariz), representó el
orígen de un modelo cinematográfico que sería tomado y explotado con fines económicos durante la próxima década. %citation

Sin embargo, su contexto no nos permite conocer a fondo la recepción crítica y
popular de la obra, debido al casi simultáneo estreno de \textit{María
  Candelaria} que eclipsó  a \textit{Flor Silvestre} %citation needed
De la misma manera, como la obra tiene un carácter melodramático, el espacio que ofrece para comentar sobre el movimiento reovlucionario es limitado, ya que cumple una función secundaria en el desarrollo de los acontecimientos de la obra.
 
(199 palabras)
