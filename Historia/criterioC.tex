\section{Evaluación de Fuentes}
% Ideal: Entre 200 y 400 palabras. Total:
\subsection{Enamorada}
La primera película que discutiremos es \textit{Enamorada}\autocite{fernandez_enamorada_1946}, por el director mexicano Emilio Fernández, y lanzada en el año de 1946 por Panamerican Films, S.A. La película cuenta con fotografía del reconocido artista mexicano Gabriel Figueroa y las actuaciones estelares  de María Félix en el papel de Beatriz Peñafiel, y Pedro Armendáriz, como el general Juan José Reyes. 

La trama es una libre adaptación de la comedia de Shakespeare \textit{La fierecilla domada} y trata sobre la llegada del general Reyes (personaje ficticio) a Cholula puebla, el sitio revolucionario que establece allí y su eventual enamoramiento hacia Beatríz Peñafiel, que pone a prueba su paciencia y honra. \textit{Grosso modo,} la película no aborda con profundidad temas políticos o controversiales, podemos considerar que su propósito principal era ofrecer una representación idealizada del periodo revolucionario, para apelar y gustar al público general y garantizar que fuera una inversión segura, pero no completamente carente de arte. 

La película es importante puesto que es considerada como una de las obras mayores del director, que como vimos, es de los principales promotores del cine mexicano, y en general, de las figuras culturales más importantes de México. La película fue tan influyente, que tres años después se realizó una adaptación Hollywoodense bajo el título \textit{Del odio nace el amor}, por el mismo director, pero naturalmente, de mucho menor calidad. Además, cuenta con un equipo cinematográfico consolidado por otros grandes artistas, como Gabriel Figueroa, considerado como el mejor fotógrafo en el la esfera cinematográfica mexicana, así como los icónicos actores María Félix y Pedro Armendáriz. Esto significa, que es un buen indicio de lo que se consideraba en aquel entonces como la visión popular u oficial del periodo revolucionario. 

Sin embargo, es el hecho mismo de la naturaleza económica de las obras cinematográficas que en este caso actúa como limitación, es decir, la intervención de productores e inversionistas cuyos capital se encuentra detrás de la obra, nos impide confiar que esa sea efectivamente la visión popular de la revolición y no completamente una representación utópica con fines exclusivamente económicos.

(355 palabras)
\subsection{La cucaracha}
Falta
