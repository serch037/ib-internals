\section{Evaluación de Fuentes}
% Ideal: Entre 200 y 400 palabras. Total:
\subsection{Flor Silvestre}
Una obra a analizar será \textit{Flor Silvestre}, del director mexicano Emilio ``el Indio'' Fernández. 
Fue producida y estrenada en 1943\autocite[17]{garcia_riera_historia_1992}.
% Cuenta con fotografía de Gabriel Figueroa y las actuaciones de Pedro Armendáriz y Dolores del Río\autocite[16]{garcia_riera_historia_1992}. 
Fue producida por Films Mundiales\autocite[17]{garcia_riera_historia_1992}, una de las productora con una agenda explícitamente nacionalista.\autocite[10]{garcia_riera_historia_1992} 
Es considerada como ``el melodrama revolucionario por excelencia dentro del cine
nacional''\autocite[178]{guerrero_imagen_2005},
y fue el  primer exito del director\autocite[19]{costa_cine_2010}.

Su intención primordial, es generar un beneficio económico.
Sin embargo, dada la ideología del autor, otra de sus intenciones es ``ofrecer una versión de México y de la revolución''%\todo{interview}
, pues pretende ser ``eminentemente didáctica y escolarmente aleccionadora''\autocite[28]{blanco_aventura_1993}.

La película, al situarse durante la lucha revolucionaria y tratar sobre un revolucionario nacido de una familia conservadora, nos permite observar la manera en la que el nacionalismo del director sintetizó al movimiento y sus causas sociales a la vez que lo reinterpreta bajo la óptica de su contexto político. Además, fue estrenada en un momento clave para la historia del cine dado que es considerado ``el gran año''\autocite[7]{garcia_riera_historia_1992} y como resultó ser el primer éxito del director, fijó un modelo que sería repetido en el resto de sus películas\autocite[86]{demello_unfinished_2001}.

Sin embargo, al ser recibir financiamiento externo, no es sencillo separar la visión del director de aquella de sus productores. 
Además, como la obra tiene un carácter melodramático, el espacio que ofrece para reflexionar sobre la revolución es limitado, pues cumple una función secundaria en el desarrollo de la trama.\autocite[365]{sanchez_vi._2010}

%(262 palabras)

\subsection{Enamorada}
Esta fuente fue producida y estrenada en 1946  por \emph{Panamericana Films} y Benito Alazraki.
Durante ese año el cine nacional experimentó una importante fase de transición, ya que el fin de la segunda guerra mundial expuso al cine mexicano hacia mayores audiencias.\autocite[12]{garcia_riera_historia_1993}
Dirigida por Emilio Fernández\autocite[59]{garcia_riera_historia_1993}, ganó ocho premios \emph{Ariel}\footnote{Otorgado por la academia mexicana de ciencias y artes cinematográficas}. 

Como es una producción cinematográfica su propósito principal es generar una solvencia y ganancia económica, sin embargo, al formar parte del cine nacionalista de la época, cumple también la función de elogiar la figura del revolucionario y ofrecer sus respetos a la Iglesia para exaltar un sentimiento nacionalista en el espectador\autocite[60]{garcia_riera_historia_1993}. 

Su valor recae en que permite al espectador obtener una perspectiva sobre cómo era entendida o cómo quería recordarse al movimiento revolucionario de manera retrospectiva, particularmente en su defensa de la figura del revolucionario. Además, ya que recibió múltiples premios nacionales, podemos inferir que refleja la visión popular. 

Sin embargo, como la obra pertenece al género melodramático, el énfasis que hace en el movimiento revolucionario es pequeño a comparación de la trama romántica entre los protagonistas. Además, como es una producción conjunta, a pesar de ser \emph{cine de autor}, no podemos aislar completamente la visión del director.  

%(212 palabras)
%(473 palabras)
% \subsection{Río escondido}
% Otra importante obra es \textit{Río escondido}, igualmente del Director mexicano Emilio ``el Indio'' Fernández. Esta fue producida en 1947\autocite[143]{garcia_riera_historia_1993} 
% , el ``primer año del sexenio del presidente Miguel Alemán''\autocite[105]{garcia_riera_historia_1993}
% , y estrenada en 1948\autocite[144]{garcia_riera_historia_1993}
% . Cuenta con las actuaciones estelares de María Félix y Fernando Fernández\autocite[143]{garcia_riera_historia_1993} 
% . La producción estuvo a cargo de Raul de Anda\autocite[143]{garcia_riera_historia_1993} 
% . Cabe mencionar que fue la única película de ese año ``que ganó premios internacionales''\autocite[108-109]{garcia_riera_historia_1993}.

% Debido a que se trata de una inversión económica, su principal intención es generar una ganancia en esos mismos términos. Sin embargo, entre las intenciones discursivas del director se presenta su interés y valoración por la educación en México, específicamente, a partir de la ``representación  de la situación de los maestris rurales''\autocite[72]{tierney_myths_2002}
% , además de ser considerada aquella que ``mejor revela los ideales del director''\autocite[171]{mora_mexican_1978-2}.

% Resulta además valiosa por la crítica polarizante que inspiró\autocite[170-171]{mora_mexican_1978-2}
% ya que tanto izquierdistas como derechistas la criticaron por razones opuestas\autocite[144]{garcia_riera_historia_1993}
% , lo cual ofrece un panorama de las contradicciones en el cine de la época.
% A pesar de esto, recibió una variedad de premios\autocite[108-109]{garcia_riera_historia_1993}

% Sin embargo, debido a que en ese mismo año el presidente establece nuevas medidas de producción estandarizada\autocite[105]{garcia_riera_historia_1993}
% , las libertades que puede tomar el director para hacer un comentario crítico son aún menores.

(428 palabras)
\pagebreak

