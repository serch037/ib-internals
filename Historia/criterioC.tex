\section{Evaluación de Fuentes}
% Ideal: Entre 200 y 400 palabras. Total:
(236 palabras)
\subsection{Flor Silvestre}
La primera obra a analizar es \textit{Flor Silvestre}, del director mexicano
Emilio ``el Indio'' Fernández. Fue estrenada en 1943. Cuenta con fotografía de
Gabriel Figueroa y las actuaciones de Pedro Armendáriz y Dolores del Río. 
Es considerada por algunos críticos como la obra maestra de este equipo de prodcucción %citation Ayala

Esencialmente, la obra es una comedia romántica con la revolución como trasondo, cuya intención es ``''%citation about movie  
, por lo tanto también presenta al revolucionario como un hombre virtuoso y respetable.
Es valiosa por ser considerada como la única producción melodramática que
entabla una reflexión sobre el movimiento revolucionario. %citation needed
Además, por su naturaleza como producción conjunta nos permite observar la
manera en la cuál la ideología del director se entrecruza con la de su contexto. 
Finalmente, fue estrenada en un momento clave para la historia del cine
% citation 
y por su equipo de producción, representó el
orígen de un modelo cinematográfico que sería tomado y explotado con fines económicos durante la próxima década. %citation

Sin embargo, su contexto no nos permite conocer a fondo la recepción crítica y
popular de la obra, debido al casi simultáneo estreno de \textit{María
  Candelaria} que eclipsó  a \textit{Flor Silvestre} %citation needed
De la misma manera, como la obra tiene un carácter melodramático, el espacio que ofrece para comentar sobre el movimiento reovlucionario es limitado, ya que cumple una función secundaria en el desarrollo de los acontecimientos de la obra.
 
(199 palabras)

\subsection{}
