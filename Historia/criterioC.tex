\section{Evaluación de Fuentes}
% Ideal: Entre 200 y 400 palabras. Total:
\subsection{Flor Silvestre}
Una obra a analizar será \textit{Flor Silvestre}, del director mexicano Emilio ``el Indio'' Fernández. 
Fue estrenada en 1943. Cuenta con fotografía de Gabriel Figueroa y las actuaciones de Pedro Armendáriz y Dolores del Río. Fue producida por Films Mundiales, una de las productoras ``alentadas por el Banco Nacional Cinematográfico \ldots interesados por un cine nacionalista''. %\todo{riera}

Su intención primordial, debido a que se trata de una obra filmográfica, es generar un beneficio económico.
Sin embargo, dada la ideología del autor, otra de sus intenciones es ``ofrecer una versión de México y de la revolución Mexicana''%\todo{interview}
, con fines tales que pretende ser ``eminentemente didáctica y escolarmente aleccionadora''. %\todo{ayala pesadilla}

Es valiosa por ser considerada como ``el melodrama melodrama revolucionario por excelencia dentro del cine
nacional'' %\todo{ref mujeres}
así como su primer éxito. %\todo{costa-s 19}
Además, por su naturaleza como producción conjunta nos permite observar la manera en la cuál la ideología del director se entrecruza con la de su contexto. 
Finalmente, fue estrenada en un momento clave para la historia del cine dado que es considerado ``el gran año''. %\todo{citation riera}
Además, es la primer película filmada con el elenco que sería considerado es el primero ``de lujo del cine mexicano''%\todo{riera flor}
, que fijaría el modelo del resto de sus películas.%\todo{rev inter 86}

Sin embargo, debido a que recibe financiamiento indirecto del Banco Cinematográfico, %\todo{ref producción de clasa}
se ve influenciado por las preferncias ideológicas del mismo, y por lo tanto no es sencillo separar la visión del director de aquella de sus productores. 
De la misma manera, como la obra tiene un carácter melodramático, el espacio que ofrece para comentar sobre el movimiento reovlucionario es limitado, ya que cumple una función secundaria en el desarrollo de los acontecimientos de la obra.%\todo{cita telón}
 
%(263 palabras)

\subsection{Río escondido}
Otra importante obra es \textit{Río escondido}, igualmente del Director mexicano Emilio ``el Indio'' Fernández. Esta fue producida en 1947 %\todo{Riera} 
, el ``primer año del sexenio del presidente Miguel Alemán''
, y estrenada en 1948 %\todo{Riera }
. Cuenta con las actuaciones estelares de María Félix y Fernando Fernández %\todo{Riera }
. La producción estuvo a cargo de Raul de Anda %\todo{Riera }
. Cabe mencionar que fue la única película de ese año ``que ganó premios internacionales''. %\todo{Riera }

Debido a que se trata de una inversión económica, su principal intención es generar una ganancia en esos mismos términos. Sin embargo, entre las intenciones discursivas del director se presenta su interés y valoración por la educación en México, específicamente, a partir de la ``representación  de la situación de los maestris rurales'' %\todo{mitos 72}
, además de ser considerada aquella que ``mejor revela los ideales del director''todo{eference mora 171}

Resulta además valiosa por la crítica polarizante que inspiró %\todo{historiag 171}
ya que tanto izquierdistas como derechistas la criticaron por razones opuestas %\todo{riera}
, lo cual ofrece un panorama de las contradicciones en el cine de la época.
A pesar de esto, recibió una variedad de premios %\todo{historia g and riera}

Sin embargo, debido a que en ese mismo año el presidente establece nuevas medidas de producción estandarizada%\todo{riera }
, las libertades que puede tomar el director para hacer un comentario crítico son aún menores.

\pagebreak
%(460)
