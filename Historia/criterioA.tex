\section{Plan de Investigación}
% Ideal: Entre 100 y 150 palabras. Total: 178 (+28)
Esta investigación se propone resolver a la pregunta,
¿De qué manera presenta el cine del Indio Fernández las causas y objetivos sociales de la revolución? Específicamente, el estudio abarcará los periodos de 1943 a 1946, años que marcaron el surguimiento y apogeo\footnote{En cuanto a su relevancia cultural} de la obra del director\autocite[133]{mora_mexican_1978-2}.

La pregunta es relevante dado que corresponde al peiodo conocido como la ``Época de Oro'' del cine mexicano, durante la cuál el cine se consolidó como una industria y un medio de cultura nacional e internacionalmente relevante, proceso en el cuál el director seleccionado jugó un papel relevante\autocite[133]{mora_mexican_1978-2}.

Un método a utilizar será el análisis del mensaje a través de la presentación de los temas sociales de: diferencia de clases, relación con la iglesia y naturaleza del movimiento en \emph{Flor Silvestre} y  \emph{Enamorada}, dos películas contextualizadas en la lucha revolucionaria. Además, se compararán y contrastarán interpretaciones críticas de distintos autores para profundizar en el análisis del mensaje.  Asímismo, se usarán datos biográficos del director y del contexto político como herramientas complementarias para la interpretación. 

%(171 palabras)
\pagebreak
