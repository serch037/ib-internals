\section{Plan de Investigación}
% Ideal: Entre 100 y 150 palabras. Total: 164 (+14)
Esta investigación se propone resolver a la pregunta,
¿De qué manera es representado México en las películas de Emilio Fernández? 
La pregunta es importante porque el estudio del cine, dada la importancia
contextual del medio, nos permite conocer sobre su contexto histórico-cultural y
político.  %cita monsiváis
Notablemente, el tema de la revolución es particularmente importante en el proceso de creación de identidad del mexicano. %cita ¿?

El periodo de estudio abarcará de 1943 a 1949, años que marcan el surguimiento, clímax y apogeo de la obra fílmica del director con su equipo de producción. %citas
Un método a utilizar será el análisis de las películas más representativas del director producidas o lanzadas durante el periodo seleccionado; la películas serán escogidas por su relevancia histórica y por su reconococimiento nacional e internacional. Además, se hará uso de comentarios por críticos y otras fuentes secundarias para contextualizar las películas y observar la manera en que responden a las necesidades de la época a la que pertenecen.

(164 palabras)
\pagebreak
