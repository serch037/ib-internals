\section{Plan de Investigación}
% Ideal: Entre 100 y 150 palabras. Total: 168 (+18)
Esta investigación se propone resolver a la pregunta,
¿Cuál fue la perspectiva popular transmitida por el cine mexicano sobre la revolución mexicana tras la posguerra partir de dos películas representativas? Es importante porque por medio de un medio tan popular como lo fue en el cine de esa época, podemos deducir la perspectiva dominante en México sobre el más significativo evento de su historia reciente, la revolución mexicana, en un momemento particularmente crítico para la identidad nacional como lo fueron los años de la posguerra.

La investigación se enfocará en el periodo entre 1946 y 1959, el inicio de la posguerra y el fin de la época de oro del cine mexicano, respectivamente. Para ella, se analizarán las representaciones revolucionarias en las películas \textit{Enamorada} -- 1946 y \textit{La cucaracha} -- 1958 escogidas por su, temática revolucionaria, su significancia cinematográfica, y similar equipo, pero distinta fecha de producción, lo cual las vuelve idóneas para los propósitos de esta investigación. Además, se hará uso de críticas y fuentes secundarias para contextualizarlas históricamente.

(168 palabras)
