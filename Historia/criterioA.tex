\section{Plan de Investigación}
% Ideal: Entre 100 y 150 palabras. Total: 170 (+20)
Esta investigación se propone resolver a la pregunta,
¿En qué medida y se distinguen los mensajes respecto a los sacrificios del revolucionario en las películas sobre la revolución
mexicana de Fernando de Fuentes y Emilio el Indio Fernández hasta 1943? 
La pregunta es importante porque el estudio del cine, dada la importancia
contextual del medio, nos permite conocer sobre el contexto histórico-cultural y
político de las obras.  %cita monsiváis
Notablemente, el tema de la revolución es particularmente importante en el proceso de creación de identidad del mexicano. %cita ¿?
Los directores fueron elegidos porque distinguidos críticos consideran a estos dos directores como los más importantes de sus respectivos periodos. %citas

El alcance de la investigación se enfocará en los años entre 1936 y 1943, años
clave para la industria cinematográfica en México, el primero por ser la fecha de inicio de la ``Época de Oro'' %citation needed
, y el segundo por ser considerado ``el año del cine''. 
Un método a utilizar será el análisis de una película representativa de cada uno
de los directores y la relación del mensaje con el periodo; la película será
escogida por su relevancia histórica y por su reconococimiento.
Estás serán: ¡Vámonos con Pancho Villa! y Flor silvestre. %cita de cada una
Además, se hará uso de comentarios por críticos y otras fuentes secundarias para contextualizar a las películas y observar la manera en que responden a las distintas necesidades de sus épocas.

(223 palabras)
\pagebreak
