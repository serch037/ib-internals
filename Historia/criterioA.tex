\section{Plan de Investigación}
% Ideal: Entre 100 y 150 palabras. Total: 164 (+14)
Esta investigación se propone resolver a la pregunta,
¿En qué medida la representación de México en las películas del Indio Fernández favorece al estado posrevolucionario? Específicamente, el estudio abarcará los periodos de 1943 a 1949, años que marcaron el surguimiento, apogeo y ocaso\footnote{En cuanto a su relevancia cultural} de la obra del director.%\todo{ref}

La pregunta es relevante dado que corresponde al peiodo conocido como la ``Época de Oro'' del cine mexicano durante la cuál el cine de México se consolidó como una industria y un medio de cultura nacional e internacionalmente relevante, proceso en el cuál el director elegido jugó un papel relevante.%\todo{mora 133}

Un método a utilizar será el análisis de la presentación de los temas sociales en \emph{Flor Silvestre} y  \emph{Río escondido}, dos películas representativas del director producidas, ya que obtuvieron reconocimientos nacional e internacionalmente, asímismo, se hará un uso ocasional de otras de sus películas para observar temas comunes. Además, se compararán y contrastarán interpretaciones críticas de distintos autores y se utilizarán fuentes secundarias para contextualizar las películas y observar la manera en que se relacionan con su contexto sociopolítico. 

%%(169 palabras)
\pagebreak
