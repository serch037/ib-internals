\section{Análisis}
En primera instancia, dada la similar temática y el breve período que separa a las obras podría (media década) uno esperar que discutan el tema de la revolución de manera similar, sin embargo, este no es el resultado observado.
Las políticas del país, así como la industria cinematográfica sufrieron grandes cambios en este lapso. 

Por ello se insiste en que considerar el mensaje de las obras --en realidad de cualquier producción cultural--, es esencial observar la correspondencia que entablan con sus respectivos contextos políticos y culturales. 
Es evidente de que ambas obras provienen de ambientes muy distintos. \textit{¡Vámonos con Pancho Villa!} fue realizada durante un periodo de mucha mayor libertad artística %citation needed
, por no mencionar que el propio director es considerado como un caso excepcional dentro del cánon cinematográfico mexicano por su honesta representación de los acontecimientos. %citation needed
Por otro lado, \textit{Flor Silvestre} no solo fue realizada durante en el cual el propio gobierno había llamado a la unidad nacional y pronto ejercería su influencia en la industria cinematográfica, sino que el propio director tiene una tendencia nacionalista y de orden didáctico dada su adopción de ideales vasconcelistas. 
Ahora bien, esta última afirmación tiene un importante matiz: a pesar de que algunos académicos consideran que el cine de Emilio Fernández contiene elementos de crítica social, \textit{Flor Silvestre} no profundiza en esa cuestión. 
De hecho, otros críticos mencionan que  \textit{Flor Silvestre}, así como el resto de las películas de la Época de Oro, el comentario social y aquel relativo a la revolución es nulo. %citation needed


\textit{¡Vámonos con Pancho Villa!}, por medio de la historia de un grupo de seis hombres autodenominado \textit{los Leones de San Pablo} que se une al movimiento villista, le plantea al observador la cuestión de hásta qué punto los sacrificios que hicieron los revolucionarios fueron fructuosos y de las verdaderas motivaciones de los revolucionarios. 
A pesar de que la película no los muestra directamente mediante una representación del México posrevolucionario, sí escenifica una secuencia de muertes absurdas y crueles. Cada uno de los Leones fallece de maneras progresivamente menos necesarias y los resultados que sus esfuerzos brindan parecen ser inexistentes. 
Por ejemplo, mientras que los primeros tres mueren en combate --aunque el último de estos cae por fuego amistoso--, los siguientes mueren a manos del propio Tiburcio, líder moral del grupo; el primero de estos fallece en un juego similar a la ruleta rusa, y el segundo es ``sacrificado'' por haberse contagiado de viruela. Si consideramos el final alternativo, entonces la cuestión se problematiza incluso más, ya que el Tiburcio es asesinado por los propios villistas, después de protestar por la implacable decisión de Villa por masacrar a la esposa e hija de Tiburcio para que este no tuviera compromisos que le impidieran regresar al movimiento villista, despúes de haber sido abandonado por ellos en el desierto. 

\textit{Flor Silvestre} también reconoce estos inmensos sacrificios que algunos revolucionarios tuvieron que realizar por su causa, sin embargo desde el momento en que comienza, deja en claro que llevaron a un México objetivamente mejor y bajo toda consideración,  más justo.%citation needed
Esperanza, representada por Dolores del Río, le cuenta a su hijo en un presente posrevoluconario sobre las precarias condiciones antes de la revolución y los sacrificios de hombres como su padre para mejorar a México. Entonces se desenvuelve el resto de la trama como un recuerdo desde ese presente seguro. Para \textit{Flor Silvestre}, a diferencia de \textit{¡Vámonos con Pancho Villa!}, la revolución fue definitivamente un proceso que brindó mejores resultados para el pueblo y los involucrados, a pesar de los sacrificios que pudo haber costado. 

Otra cuestión que ambas películas desarrollan es el tema de la crueldad. \textit{¡Vámonos con Pancho Villa!} inicia con un breve comentario del director explicando cómo a pesar de que algunos eventos de la película pueden parecer inhumanamente crueles, no es prudente juzgar a las figuras involucradas en términos de si fueron buenos o malos. 
En cambio, reconoce la complejidad de figuras como Pancho Villa que cometieron actos tanto generosos, como violentos y aparentemente crueles.
Bajo esta óptica, podemos interpretar la primera y última aparición del líder revolucionario como una representación de la dualidad moral del personaje. En la primera, se le ve repartiendo maíz a los campesinos mientras les dice que la revolución les permitirá dejar de ser peones y ser libres como individuos, por otro lado, en su última aparición, tanto en el final alternativo como en el original, se le ve abandonando  sin reconocimiento alguno a quienes incicialmente lucharon por su causa pero han dejado de serle útiles.

En contraparte, a pesar de reconocer la violencia ejercida durante el movimiento revolucionario, \textit{Flor Silvestre } sugiere que esta provino no de auténticos revolucionarios, sino de bandidos oportunistas, en la película representados por los hermanos que asesinan al padre del héroe y toman su hacienda por la fuerza. Para esta película, la revolución fue internamente un moviemiento unificado y noble, ya que solo buscaba la justicia. De hecho, contaba con el consentimiento y cooperación de la propia iglesia --hemos de recordar que Ávila Camacho se proclamó como el primer presidente católico tras la reovlución--%citation.
(conteo de palabras 845)
