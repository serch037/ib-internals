\section{Análisis}
Esta investigación es importante en su contexto histórico debido a que 
\hlc[yellow]{
el cine se consolidó como una industria nacionalmente relevante
}
\autocite[7]{garcia_riera_historia_1992}
, por lo que podemos considerar que participó, 
\hlc[green]{
así como afirma Escobar, 
}
\autocite[12]{silva_escobar_epoca_2011}
en el
\hlc[red]{
proceso de creación de una identidad nacional patrocinado por el gobierno
}
\autocite[35]{tierney_myths_2002}.
En términos generales, 
\hlc[blue]{
las obras cinematográficas de este periodo destacaron por su conservadurismo
}
\autocite[179]{guerrero_imagen_2005}
y alineación al discurso oficial, 
ya que el aparato nacional de censura, 
\hlc[magenta]{
prohibía un cine crítico-histórico
}
\autocite[179]{guerrero_imagen_2005},
\hlc[Apricot]{
Este discurso oficial tendía promover la  unidad nacional entre las distintas clases y sectores sociales
}
\autocite[365-366]{sanchez_vi._2010}, 
motivo por el cuál, como indica Monsiváis 
\hlc[Bittersweet]{
ocurrió una transformación en la interpretación histórica popular, que en vez de proponerse relatar los acontecimientos del pasado de manera realista, los fragmentase y recuperase las piezas que le fuese de mayor utilidad. 
}
\autocite[440]{cosio_villegas_notas_1976}
Por este motivo,
\hlc[Mahogany]{
el cine de la época tendió a realizar interpretaciones convenientemente revisionistas de la revolución
}
\autocite[370]{sanchez_vi._2010}.

El director Emilio ``El Indio'' Fernández participó en este proceso,
\hlc[DarkOrchid]{
al grado de servir como un representante ideal del  movimiento de nacionalismo cultural
}
\autocite[445]{cosio_villegas_notas_1976},
ya que no solo fueron 
\hlc[LimeGreen]{
películas como las suyas que el gobierno buscaba incentivar
}
\autocite[10]{garcia_riera_historia_1992} 
, 
sino que él mismo buscaba que su cine cumpliera una función
\hlc[BlueViolet]{
educativa
}
\autocite[16:30]{soler_serrano_emilio_1976}
\hlc[CadetBlue]{
mediante la exaltación de un sentimiento nacionalista
}
\autocite[127]{aguilar_construccion_2014-1}
\hlc[Cerulean]{
y la incorporación de temáticas sociales en sus películas
}
\autocite[72]{tierney_myths_2002}. 
Por estas razones, hay quienes consideran que la construcción de sus películas
\hlc[OrangeRed]{
se da principalmente a través del mito
}
\autocite[445]{cosio_villegas_notas_1976}.
Sin embargo, hay quienes consideran que su cine cumple también con una 
\hlc[Melon]{
función de denuncia y crítica social
}
\autocite[466]{pablos_escuela_1998-1}
y afirman que, más que alinearse con el discurso nacionalista oficial, lo que demuestra en sus películas es 
\hlc[Mulberry]{
una visión propia de México y su futuro
}
\autocite[158]{aguilar_construccion_2014-1}.

En \emph{Flor Silvestre}, 
\hlc[RedOrange]{
la revolución es presentada como un acontecimiento violento, que costó múltiples vidas, pero cuyos sacrificios rindieron frutos para la construcción de un México más justo
}
\autocite[1:30:10]{fernandez_flor_1943}. Aquí es clara la relación que podemos hacer con parte del discurso de la época, ya que a pesar de haber sido un proceso de incontables muertes, 
\hlc[Mahogany]{
su resultado es la paz, alineándose a la generalización de Franco
}
 \autocite[370]{sanchez_vi._2010}. 
Por este motivo, Franco concluye que aunque se ambienta en la revolución, 
\hlc[Rhodamine]{
la película es más afín a su propia época, que a la lucha misma
}
\autocite[365]{sanchez_vi._2010}.
Esta interpretación es soportada también por la 
\hlc[RoyalPurple]{
escena que resume la lucha a lo largo del país, donde podemos apreciar una traslación del discurso Avila-Camachista que pide unidad nacional hacia Madero
}
\autocite[42:38-43:20]{fernandez_flor_1943}.
Así como por el hecho de que, 
\hlc[Aquamarine]{
la película ignore la guerra cristera
}
\autocite[370]{sanchez_vi._2010}
para presentar una revolución que nunca desafiase la autoridad de la Iglesia. 
Por otro lado, es importante recalcar, como hace DeMello, que \hlc[SpringGreen]{
aunque presenta ambigüedades respecto a los revolucionarios, en conclusión se justifican sus acciones
}
\autocite[87]{demello_unfinished_2001}. 
Aunque ciertos críticos, como Blanco consideren que 
\hlc[TealBlue]{
se presenta a la lucha como un proceso caótico que costaba vidas inocentes
}
\autocite[36]{blanco_aventura_1993},
\hlc[Turquoise]{
si nos apegamos a la película, esta responsabilidad es solo parcialmente relegada a la propia revolución, ya que cede esta misma a oportunistas que aprovechan la confusión latente para su propio beneficio
}
\autocite[48:07-48:40]{fernandez_flor_1943}.
Finalmente, respecto a los objetivos de la lucha, la película, por su temática, concluye que eran principalmente de 
\hlc[Rhodamine]{
igualdad entre clases
}
\autocite[365]{sanchez_vi._2010}
\hlc[CarnationPink]{
y repartición de tierras
}
\autocite[11:14]{fernandez_flor_1943}. 

\emph{Enamorada} intensifica los temas presentados en \emph{Flor Silvestre}, ya que 
\hlc[Dandelion]{
elabora una similar adjudicación de los posibles males de la revolución hacia los oportunistas que abusan de la temporal confusión 
}
\autocite[14:11]{fernandez_enamorada_1946} 
y 
\hlc[Gray]{
explícitamente niega que los revolucionarios sean bandidos
}
\autocite[5:31-5:46]{fernandez_enamorada_1946}. 
Como menciona Franco,
\hlc[MidnightBlue]{ 
también se alude a una supuesta amistad entre la Iglesia y la revolución
}\autocite[377]{sanchez_vi._2010}, 
que se hace explícita en la 
\hlc[Orange]{
amistad entre el personaje del cura y el general revolucionario
}
\autocite[4:57-5:13]{fernandez_enamorada_1946}. 
Por otro lado, la primera parte de la película es incluso más ambigua sobre los métodos de la revolución, 
al mencionar explícitamente de que 
\hlc[Periwinkle]{
no es sencillo determinar si los revolucionarios son buenos o malos}
\autocite[39:00]{fernandez_enamorada_1946}.
Esto ocurre porque
\hlc[Orchid]{
muestra el uso de la violencia para arrebatar directa e ilegalmente bienes materiales
}
\autocite[4:57-5:13]{fernandez_enamorada_1946}
, que utiliza, 
\hlc[SeaGreen]{
para financiar la escuela local
}\autocite[18:30]{fernandez_enamorada_1946} 
\hlc[SkyBlue]{
y la causa misma
}\autocite[21:38]{fernandez_enamorada_1946} 
Por otro lado, la segunda parte de la película redime esta aparente ambigüedad mostrando a los revolucionarios como hombres honrados pues \hlc[Thistle]{
devuelven algunas de estas riquezas
}
\autocite[1:27:21-1:35:20]{fernandez_enamorada_1946}
Asímismo,
\hlc[YellowOrange]{
simbólicamente muestra que los revolucionarios siempre actúan en beneficio de la nación
}
\autocite[03:04]{fernandez_enamorada_1946}.
De esta manera, aunque parece ser que en un inicio se crítica la lucha revolucionaria, su conclusión los redime de los posibles males que hayan podido causar. 

Tierney sintetiza  estas posibles ambigüedades indicando que 
\hlc[AliceBlue]{
el cine de Fernández se permite ciertas críticas a la sociedad dentro de los límites de lo respetable
}
\autocite[48]{tierney_myths_2002}
para alcanzar el objetivo final, como menciona Aguilar, 
\hlc[Mulberry]{
de construir una visión propia
}
\autocite[158]{aguilar_construccion_2014-1}. 

% Aguilar sintetiza esta posturas indicando que su obra, aunque nacionalista\autocite[127]{aguilar_construccion_2014-1}
% , más que apegarse al proyecto oficial, se forma de acuerdo a la visión propia del autor\autocite[158]{aguilar_construccion_2014-1}.

(757 palabras)
\pagebreak
