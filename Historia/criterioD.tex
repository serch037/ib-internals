\section{Análisis}
Esta investigación es importante en su contexto histórico debido a que 
\hlc[yellow]{
el cine se consolidó como una industria nacionalmente relevante
}
\autocite[7]{garcia_riera_historia_1992}
, por lo que podemos considerar que participó, 
\hlc[green]{
así como afirma Escobar, 
}
\autocite[12]{silva_escobar_epoca_2011}
en el
\hlc[red]{
proceso de creación de una identidad nacional patrocinado por el gobierno
}
\autocite[35]{tierney_myths_2002}.
En términos generales, 
\hlc[blue]{
las obras cinematográficas de este periodo destacaron por su conservadurismo
}
\autocite[179]{guerrero_imagen_2005}
y alineación al discurso oficial, 
ya que además del aparato nacional de censura, 
\hlc[magenta]{
que prohibía un cine de crítica histórica
}
\autocite[179]{guerrero_imagen_2005},
\hlc[cyan]{
el clero ejercía también una considerable influencia sobre el contenido de las mismas
}
\autocite[12]{garcia_riera_historia_1992}.
\hlc[Apricot]{
Este discurso oficial tendía promover la  unidad nacional entre las distintas clases y sectores sociales
}
\autocite[365-366]{sanchez_vi._2010}, 
motivo por el cuál, como indica Monsiváis 
\hlc[Bittersweet]{
ocurrió una transformación en la interpretación histórica popular, que en vez de proponerse relatar los acontecimientos del pasado de manera realista, los fragmentase y recuperase las piezas que le fuese de mayor utilidad. 
}
\autocite[440]{cosio_villegas_notas_1976}
Por este motivo,
\hlc[Mahogany]{
el cine histórico de la época tendió a realizar interpretaciones bastante irreales pero convenientes para el gobierno de la revolución
}
\autocite[370]{sanchez_vi._2010}.

El director Emilio ``El Indio'' Fernández participó activamente en este proceso,
\hlc[DarkOrchid]{
al grado de servir como un ejemplar ideal de los objetivos del  movimiento de nacionalismo cultural
}
\autocite[445]{cosio_villegas_notas_1976},
ya que no solo fueron 
\hlc[LimeGreen]{
películas como las suyas que el gobierno buscaba incentivar
}
\autocite[10]{garcia_riera_historia_1992} 
, 
sino que él mismo buscaba que su cine cumpliera una función
\hlc[BlueViolet]{
educativa
}
\autocite[16:30]{soler_serrano_emilio_1976}
\hlc[CadetBlue]{
mediante la exaltación de un sentimiento nacionalista}
\autocite[127]{aguilar_construccion_2014-1}
\hlc[Cerulean]{
y la incorporación de temáticas sociales en sus películas
}
\autocite[72]{tierney_myths_2002}. 
Por estas razones, hay quienes consideran que su presentación de temáticas tales como los objetivos de la revolución, o la presentación de su proceso 
\hlc[OrangeRed]{
se da principalmente a través del mito
}
\autocite[445]{cosio_villegas_notas_1976}.
A pesar de esto, hay quienes consideran que su cine cumple también con una 
\hlc[Melon]{
función de denuncia y crítica social
}
\autocite[466]{pablos_escuela_1998-1}
y afirman que, más que alinearse con el discurso nacionalista oficial, lo que demuestra en sus películas es 
\hlc[Mulberry]{
una visión propia de México y su futuro
}\autocite[158]{aguilar_construccion_2014-1}.

En \emph{Flor Silvestre}, 
\hlc[RedOrange]{
la revolución es presentada como un acontecimiento violento, que costó la vida de muchos, pero cuyos sacrificios por la justicia social indubitablemente rindieron frutos para la construcción de un México mejor
}
\autocite[1:30:10]{fernandez_flor_1943}. 
% \emph{Flor Silvestre}, cronológicamente\autocite[17]{garcia_riera_historia_1992}
% la primera de nuestra seleccíón, es, según Monsiváis, 
% una \emph{estetización fragmentada}\autocite[440]{cosio_villegas_notas_1976}
%  de la revolución, con propósitos comerciales ante todo\autocite[439]{cosio_villegas_notas_1976}.
% Es más, así como suguiere Franco esta representación es una visión que destaca por su revisionismo,\autocite[369]{sanchez_vi._2010}
%  que retrata la revolución como ``un proceso forjador de paz''\autocite[370]{sanchez_vi._2010}
% , que, como suguiere DeMello, a pesar de presentar algunas ambigüedades con respecto a los revolucionarios, les simpatize excesivamente\autocite[87]{demello_unfinished_2001}. 
% Por lo tanto, resulta apropiado que la película, producida durante el periodo \emph{Ávila-Camachista}, siguiendo el proceso de \emph{fragmentación} aludido por Monsiváis\autocite[440]{cosio_villegas_notas_1976}
% convenientemente ignore, como advierte también Franco la guerra cristera y represente a la revolución como un movimiento siempre ligado a la iglesia\autocite[370]{sanchez_vi._2010}.
% De esta misma forma, se adhiere a la versión oficial
% presentando el movimiento revolucionario --que busca el mejoramiento-- como uno unificado y eficiente\autocite[42:38-43:20]{fernandez_flor_1943}
% , de hecho la misma película hace referencia explícita al proyecto zapatista de reforma agraria\autocite[11:14]{fernandez_flor_1943}.

% Este mejoramiento, que ocurre de manera complementaria en \emph{Enamorada}\autocite[377]{sanchez_vi._2010}
% ,en conjunción, ambas películas implican la desaparición de las clases sociales\autocite[365]{sanchez_vi._2010}.
% Proceso que recuerda al llamado de Ávila Camacho hacia la unidad nacional\autocite[366]{sanchez_vi._2010}.
% Por otro lado, Guerrero interpreta que estas escenas presentan a la revolución como una amenaza\autocite[178]{guerrero_imagen_2005}
% para procurados por el nuevo régimen,
% que como comenta, tenían un carácter conservador\autocite[178]{guerrero_imagen_2005}
% .

% Según Ayala, Fernández reconoce que la revolución fue un proceso ambigüo, con buenas intenciones, pero que desencadenó el caos\autocite[36]{blanco_aventura_1993}, 
% si nos apegamos a la trama de la película, es evidente que no es la revolución \emph{auténtica} aquella que representa ese peligro, sino que son los oportunistas o \emph{falsos revolucionarios} aquellos responsables del bandidaje.\autocite[48:07-48:40]{fernandez_flor_1943}
% Nótese aquí el paralelismo con \emph{Enamorada}, que también verbaliza estas opiniones\autocite[14:11]{fernandez_enamorada_1946}\autocite[5:31]{fernandez_enamorada_1946}
% , misma en la cuál se hace explícita la unión entre la iglesia y la revolucíón\autocite[377]{sanchez_vi._2010}, así como la ambigüedad de la figura del revolucionario\autocite[39:00]{fernandez_enamorada_1946}.

% Tierney sintetiza  las posibles ambigüedades indicando que el cine de Fernández se permite ciertas críticas a la sociedad dentro de los límites de lo respetable\autocite[48]{tierney_myths_2002}.


% \emph{Río escondido}, nuestra segunda selección, ha sido un objeto igualmente grande de debate\autocite[144]{garcia_riera_historia_1993}.
% A pesar de que algunos afirman que se ambienta durante el periódo Alemanista\autocite[168]{mora_mexican_1978-2}.
% Por ejemplo, Monsiváis afirma, en palabras de Ayala, que la película alaba las políticas del periodo sirviendo como una fanática fábula milagrosa\autocite[78]{blanco_aventura_1993}.
% , en pos de infundir un sentimiento nacionalista\autocite[66]{consuelo_rangel_ley_2006}.
% Sin embargo, debemos de considerar el epígrafe de la misma, por medio de la cuál el director niega tal declaración.\autocite[00:00]{fernandez_rio_1947}
% Indistintamente, la administración del presidente cuyo rostro permanece oculto\autocite[12:15-12:40]{fernandez_rio_1947}
% se muestra resuelta a construir un México que podría llamarse ideal\autocite[12:51]{fernandez_rio_1947}.
% Resulta interesante, como señala Tuñon, que el periodo de Alemán tuviera un gasto relativamente bajo dedicado para la educación\autocite[451]{pablos_escuela_1998-1}
% cuando en la película se muestra por el contrario, un gran interés en esta área\autocite[12:27]{fernandez_rio_1947}.
% Por ello, señala que trata de un cine crítico que señala las deficiencias de México, y busca que se retomen las políticas sociales de las décadas anteriores\autocite[466]{pablos_escuela_1998-1}
% que han sido abandonadas\autocite[62]{consuelo_rangel_ley_2006}.

% Aguilar sintetiza esta posturas indicando que su obra, aunque nacionalista\autocite[127]{aguilar_construccion_2014-1}
% , más que apegarse al proyecto oficial, se forma de acuerdo a la visión propia del autor\autocite[158]{aguilar_construccion_2014-1}.

%(237 palabras)
\pagebreak
