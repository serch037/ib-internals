\section{Análisis}
La crítica en torno a las películas del director ha sido amplia y discutida. No solo eso, sino que como argumenta XXXXXXXX la propia figura del director ha adquirido un carácter casi mítico. %\todo{refernce monsi 445}

A diferencia de lo que algunos críticos como XXXXXXXX aseguran %\todo{reference}
, otros como XXXXXXXX indican que no hay una \emph{narrativa ideológica} común a las películas del Indio Fernández.%\todo{refence}
Incluso hay quienes, como XXXXXXXX que indican que su filmografía carece de reflexiones reales.%\todo{reference}
A pesar de esto, si valoramos las afirmaciones del propio directo
 ,así como detalles de su biografía%\todo{reference}
 , es de esperar que se presente algún tipo de reflexión social e incluso crítica social dentro de sus películas. 
Además, debido al hecho de que la industria cinematográfica%\todo{reference}
, así como el contexto político %\todo{reference}
 experimentaron, durante la década de los cuarenta, un proceso gradual de transformación%\todo{reference}
Es solo natural que las películas experimenten cambios, si no similares, sí sincronizados, particularmente por la adherencia, según una multiplicidad de críticos, como Monsiváis%\todo{refenece monsi444}
, o XXXXXXXX %\todo{refenece}
, yXXXXXXXX 
al proyecto nacionalista, considerando además que según Riera%\todo{reference Riera}
y  XXXXXXXX %\todo{reference out_oo2}
la crítica hístórica estava virtualmente prohibida en el cine.  

\emph{Flor Silvestre}, cronológicamente%\todo{reference}
la primera de nuestra seleccíón, es, según Monsiváis, 
como una \emph{estetización fragmentada}%\todo{reference monsiv 439-440}
 de la revolución, con propósitos comerciales ante todo%\todo{reference monsiv 439-440}
, aunque, contradictoriamente pero alineándose a ideología \emph{Vasconcelista} del director %\todo{reference}
busca presentar una versión de la revolución, opinión con la que concuerda XXXXXXXX.%\todo{refernce}
Es más, así como suguiere XXXXXXXX esta representación es una visión que destaca por su revisionismo,%\todo{luz y guerra 369}
 que retrata la revolución como ``un proceso forjador de paz''%\todo{luz y sombra 370}
por lo tanto, resulta apropiado que la película, producida durante el periodo \emph{Ávila-Camachista}, siguiendo el proceso de \emph{fragmentación} aludido por Monsiváis%\todo{reference}
convenientemente ignore, como advierte también XXXXXXXX la guerra cristera y represente a la revolución como un movimiento siempre ligado a la iglesia.%\todo{luz y guerra 370}
De esta misma forma, se adhiere a la versión oficial%\todo{reference}
 presentando  el movimiento revolucionario --que busca el mejoramiento-- como uno unificado y eficiente.%\todo{reference película}

Este mejoramiento, que ocurre de manera complementaria en \emph{Enamorada} %\todo{reference}
, implica la desaparición de las clases sociales. %\todo{reference}.
Por otro lado, XXXXXXXX opina que la película representa a la revolución como una amenaza %\todo{reference out_002 178}
para los valores que de acuerdo con XXXXXXXX 
son procurados por el nuevo régimen. %\todo{refernce}
XXXXXXXX comenta además sobre las tendencias conservadoras de la nueva administración %\todo{reference out_002 179}
, sin embargo la misma película contiene escenas que hacen referencia al proyecto zapatista de reforma agraria. %\todo{reference película}
Además, si nos apegamos a la trama de la película, es evidente que no es la revolución \emph{auténtica} aquella que representa ese peligro, sino que son los oportunistas o \emph{falsos revolucionarios} aquellos responsables del bandidaje. %\todo{reference película}
Nótese aquí el paralelismo con \emph{Enamorada}, que verbaliza esta opinión %\todo{refernece película}
, misma en la cuál se hace explícita la unión entre la iglesia y la revolucíón.%\todo{referene luz y guerra}



\emph{Río escondido}, nuestra segunda selección, ha sido un objeto igualmente grande de debate. 
A pesar de que algunos afirman que se ambienta durante el periódo Alemanista %\todo{reference mora}
, debemos de considerar el epígrafe de la misma, por medio de la cuál el director niega tal declaración.%\todo{reference película}
Indistintamente, la administración del presidente cuyo rostro permanece oculto%\todo{reference película}
se muestra resuelta a construir un México que podría llamrse ideal.%\todo{refernce pelícila}
Resulta interesante, como señala Tuñon, que el periodo de Alemán tuviera un gasto relativamente bajo dedicado para la educación %\todo{citation tuñon 451}
cuando en la película se muestra por el contrario, un gran interés en esta área. %\todo{reference película}
Asímismo, Tuñon señala que la problemática de la película tiene orígen en el retraso del pueblo para acoplarse al nuevo régimen%\todo{reference tuñon 458}

%325
\pagebreak
