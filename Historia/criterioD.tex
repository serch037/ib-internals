\section{Análisis}
La crítica en torno a las películas del director ha sido amplia y discutida. No solo eso, sino que como argumenta XXXXXXXX la propia figura del director ha adquirido un carácter casi mítico. %\todo{refernce monsi 445}

Si valoramos las afirmaciones del propio director%\todo{reference entrevista}
 ,así como opiniones de críticos como Julia Tuñon%\todo{reference}
 , es de esperar que se presente algún tipo de reflexión social e incluso crítica social dentro de sus películas. 
Otros, como Monsiváis%\todo{refenece monsi444}
, o XXXXXXXX %\todo{Ayala ??}
suguieren que Fernández se apega con creces 
al proyecto nacionalista, ya sea por convicción o por que, como indica Riera%\todo{reference Riera}
y  XXXXXXXX %\todo{reference out_oo2}
la crítica hístórica estava virtualmente prohibida en el cine. De acuerdo con estos críticos, la función principal del director, era participar en la creación de ``lo mexicano'', para instalar una identidad en el imaginario social nacional e internacional. %todo refenrence 694 12
Por lo tanto, así como recuerda XXXX, el gobierno mexicano tendría un importante papel en el desarrollo de la industria cinematográfica y su proyección de México. %mitos 35

\emph{Flor Silvestre}, cronológicamente%\todo{reference}
la primera de nuestra seleccíón, es, según Monsiváis, 
una \emph{estetización fragmentada}%\todo{reference monsiv 439-440}
 de la revolución, con propósitos comerciales ante todo%\todo{reference monsiv 439-440}
Es más, así como suguiere XXXXXXXX esta representación es una visión que destaca por su revisionismo,%\todo{luz y guerra 369}
 que retrata la revolución como ``un proceso forjador de paz''%\todo{luz y sombra 370}
, que, como suguiere XXXXX, a pesar de presentar algunas ambigüedades con respecto a los revolucionarios, les simpatize excesivamente.%\todo{rev_inter 88}
Por lo tanto, resulta apropiado que la película, producida durante el periodo \emph{Ávila-Camachista}, siguiendo el proceso de \emph{fragmentación} aludido por Monsiváis%\todo{reference}
convenientemente ignore, como advierte también XXXXXXXX la guerra cristera y represente a la revolución como un movimiento siempre ligado a la iglesia.%\todo{luz y guerra 370}
De esta misma forma, se adhiere a la versión oficial%\todo{reference}
presentando  el movimiento revolucionario --que busca el mejoramiento-- como uno unificado y eficiente.%\todo{reference película}
, incluso la misma película contiene escenas que hacen referencia al proyecto zapatista de reforma agraria. %\todo{reference película}

Este mejoramiento, que ocurre de manera complementaria en \emph{Enamorada} %\todo{reference}
,en conjunción, ambas películas implican la desaparición de las clases sociales. %\todo{reference}.
Por otro lado, XXXXXXXX interpreta que estas escenas presentan a la revolución como una amenaza %\todo{reference out_002 178}
para los valores que de acuerdo con XXXXXXXX 
son procurados por el nuevo régimen,
que como XXXXXXXX comenta, tenían un carácter conservador  %\todo{reference out_002 179}
.

Según Ayala, Fernández reconoce que la revolución fue un proceso ambigüo, con buenas intenciones, pero que desencadenó el caos, %\todo{ayala 36}
si nos apegamos a la trama de la película, es evidente que no es la revolución \emph{auténtica} aquella que representa ese peligro, sino que son los oportunistas o \emph{falsos revolucionarios} aquellos responsables del bandidaje. %\todo{reference película}
Este importante matiz es explicado por XXXX, quien indica que son precisamente estos quienes actúan de manera injusta.
Nótese aquí el paralelismo con \emph{Enamorada}, que verbaliza esta opinión %\todo{refernece película}
, misma en la cuál se hace explícita la unión entre la iglesia y la revolucíón.%\todo{referene luz y guerra}

XXXXXX sintetiza  las posibles ambigüedades indicando que el cine de Fernández se permite ciertas críticas a la sociedad dentro de los límites de lo respetable.%mitos 48


\emph{Río escondido}, nuestra segunda selección, ha sido un objeto igualmente grande de debate.%tofo{refernce ayala}
A pesar de que algunos afirman que se ambienta durante el periódo Alemanista %\todo{reference mora}
Por ejemplo, Monsiváis afirma en palabras de Ayala, que la película alaba las políticas del periodo sirviendo como una fanática fabula milagrosa%\todo{ayala 79}
, con el propósito de infundir un sentimiento nacionalista%.\todo{desmitificando 6}
Sin embargo, debemos de considerar el epígrafe de la misma, por medio de la cuál el director niega tal declaración.%\todo{reference película}
Indistintamente, la administración del presidente cuyo rostro permanece oculto%\todo{reference película}
se muestra resuelta a construir un México que podría llamrse ideal.%\todo{refernce pelícila}
Resulta interesante, como señala Tuñon, que el periodo de Alemán tuviera un gasto relativamente bajo dedicado para la educación %\todo{citation tuñon 451}
cuando en la película se muestra por el contrario, un gran interés en esta área. %\todo{reference película}
Asímismo, Tuñon señala que la problemática de la película tiene orígen en el retraso del pueblo para acoplarse al nuevo régimen%\todo{reference tuñon 458}
, por esta razón insiste en que se trata de un cine crítico ya que señala las deficiencias de México, y busca que el gobierno retome las políticas sociales de las décadas anteriores.%todo{ref tuñon 446}
que han sido abandonadas.%todo{ref desmitificando 2}

Aguilar sintetiza esta posturas indicando que su obra, aunque nacionalista%emilioméxico
, más que apegarse al proyecto nacionalista oficial, se forma de acuerdo a la visión propia del autor. % emiliomexico 158
%325
%691
\pagebreak
