% Estímulo ¿De qué recursos se vale el autor para desarrollar los temas? 
% Título: La topología psicológica en \emph{El rumor del oleaje} de Yukio Mishima
Yukio Mishima es considerado uno de los más importantes escritores japoneses. 
%Su vida se marcó por múltiples contradicciones, así como por un intento de encontrar un punto de equilibrio entre los puntos opuestos que lo marcaban.%FIXME ¿referencia? no
Su novela, \emph{El rumor del oleaje}, por su presentación del \emph{Locus amoenus}, nos permite percibir cómo Mishima se imaginaba un estilo de vida absuelto de las dicotomías originadas en, en equlibrio ideal con la naturaleza, mediante la idealización de la masculinidad que representa el personaje protagónico de Shinji.%reestructurar orden

Previo al análisis valdría la pena discutir brevemente sobre el orígen de la novela y el contexto biográfico del autor. 
La novela, publicada en 1954, es una libre adaptación de una obra clásica titulada \emph{Dafnis y Cloe} sobre la educación sexual de dos jovenes. %FIXME: referencia, no, nada introductorio, relacionar con tópico 
Asímismo, fue este viaje según el cuál, la estoica y físicamente virtuosa apariencia de la estatua \emph{El auriga de Delfos}, lo llevo a reestructurar su vida alrededor del paradigma del cuerpo, en contraposición a la vida de la mente que había llevado hasta entonces a costa de la humillación producida por su débil aspecto. Aunado al hecho de en aquel contexto Japón experimentaba un agudo choque cultural con occidente debido a la ocupación norteamericana, el autor, mediante su obra \emph{El rumor del oleaje} propone un regreso al idílico pasado de Japón en el cual el \emph{modus vivendi} de las personas debe procurar una profunda integración con la naturaleza para demostrar virtud, tal como hace el personaje protagónico \emph{Shinji} y en menor medida el resto de los habitantes de \emph{Uta Jima}. %explicar
 
 Respecto a esta integración ambiente-personaje inmediatamente resalta la manera en que el autor configura la trama de la novela alrededor de eventos cuyo orígen reside exclusivamente en la naturaleza. De este modo, los sucesos narrados obedecen no solo a la lógica derivada de la manera en que los personajes se relacionan entre sí, sino que también del tipo de relación que mantienen con la naturaleza.

  La localidad en que transcurren los hechos narrados, \emph{Uta Jima}, se organiza  de acuerdo a este paradigma, mediante la deificación panteísta de la naturaleza --como un reflejo del sintoísmo japonés--. %FIXME:cita
