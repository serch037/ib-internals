% Estímulo ¿De qué recursos se vale el autor para desarrollar los temas?
% Título: La topología psicológica en \emph{El rumor del oleaje} de Yukio Mishima
Yukio Mishima es considerado uno de los más importantes escritores japoneses.
%Su vida se marcó por múltiples contradicciones, así como por un intento de encontrar un punto de equilibrio entre los puntos opuestos que lo marcaban.%FIXME ¿referencia? no
Su novela, \emph{El rumor del oleaje}, publicada en 1954,  nos permite percibir cómo Mishima se imaginaba un estilo de vida íntegro y en equlibrio ideal con la naturaleza, mediante la idealización del concepto de  masculinidad representada por la formación como adulto del  personaje protagónico de Shinj.

Mishima representa a Shinji como un joven virtuoso en cuerpo y en espíritu, respetuoso y moderado en sus ambiciones, pero dispuesto a atravezar cualquier obstáculo para conseguirlas. Sus cualidades físicas --mismas que el autor tenía en alta estima en su propia vida--, tienen orígen y propósito en su pesado trabajo como pescador, ``era alto y fornido para su edad, únicamente sus facciones revelaban su juventud [\ldots] Sus ojos azules eran muy claros,[\ldots] ese don que el mar concede a quienes se ganan en él su sustento''; una labor cuyo foco se centra por definición en una estrecha relación con la naturaleza.

La novela, que es una libre adaptación del clásico griego \emph{Dafnis y Cloe}, se estructura a modo de \emph{Bildungsroman}, o novela de formación.
Por lo tanto, entre los motivos principales que encontramos en la novela, encontramos la educación sexual de los personajes principales --Shinju y Hatsue--, así como un proceso de transición desde la juventud a la adultez.
 En su primer encuentro, podemos apreciar la virtud de la inocencia en Shinji, quien en un momento casi ridículo, ``pasó a propósito por delante de ella, y de la misma manera en que los niños se quedan mirando un objeto extraño, se detuvo y la miró a la cara.''%fixreference
La experiencia le trastorna por completo, pero su candor juvenil le impide percibir que se ha enamorado  ``yació preguntándose qué le ocurría, temeroso de que aquello pudiera ser lo que la gente llamaba estar enfermo.'' %fixrefence

Por el enfoque del narrador podemos percatarnos que es Shinji aquel que realiza de manera activa la transformación a la adultez, mientras que Hatsue permanece como un personaje secundario cuya presencia es accesoria al proceso de transición de Shinji.
Para construir este proceso --y que Shinji concluya su formación como una formación idealizada de la masculinidad--, el autor le presenta numerosas pruebas físicas y morales, cada una de las cuales debe superar para demostrar su valor. Un ejemplo pertinente es el momento en que, aparentemente por accidente, yace en el mismo lecho que Hatsue, y a petición de ésta y por su falta de experiencia ignorante de las cuestiones morales relacionadas con las mujeres: ``sentía un respeto aleatorio hacia las actitudes morales. Y como aún no había conocido íntimamente a una mujer, creyó haber llegado a lo más profundo del ser de Hatsue, donde radicaba su moralidad, y no insistió más.'' limita sus avances para no ofenderla.

Esta escena, aunque breve, resulta particularmente interesante por su contenido simbólico, puesto que ilustra la falta de preparación de los jóvenes para consumar su amor.
Aquí, el autor describe la tensión erótica entre los jóvenes a partir de los elementos de la naturaleza que los rodean.
De esta manera, los eventos que aquí transcurren se desarrollan rodeados de una sensación de completa naturalidad.
Esta comienza con Shinji despertándose de una breve siesta, lo que le brinda a la escena cualidades aún más oníricas, de hecho, el propio Shinji ``se preguntó si estaba soñando''.%fixreference
Lo primero de lo que éste se percata del fuego frente a sí, que le revela la sombra de una figura desvistiéndose. El fuego, símbolo del deseo, permea el ambiente y guía a Shinji hacia Hatsue y le brinda a ésta el pretexto para desvestirse ``la idea de que se estaba desvistiendo delante de un hombre no había pasado por su mente. Sencillamente, se desvestía ante una fogata porque era el único fuego disponible y porque estaba mojada.''%¿Punto después de cita?
Shinji se aprovecha de esta situación y decide observarla por el rabillo del ojo.
No es soprendente que sea el mismo fuego que antes le presentó la figura de Hatsue aquel que delate su crimen.
Aquí, el narrador aprovecha para recordarnos que apesar de haber actuado como \emph{voyeurista}, la castidad de Shinji no ha sido perturbada: ``el honesto muchacho cerró los ojos y los apretó con fuerza. Ahora que lo pensaba, desde luego había sido un error fingir que aún estaba dormido\ldots''.
Sin embargo, tras una breve examinación de conciencia, decide abrir los ojos en un acto que resulta simultáneamente desafiante y desconcertante.
Simbólicamente, ese ``abrir de ojos'', lo encamina en su transformación como figura adulta, una libre del pudor infantil ``no comprendía que, por el mero hecho de que estuviese desnuda, se había alzado una barrera entre ellos que dificultaba las muestras de cortesía ordinarias, las confianzas naturales''.
En cambio, Hatsue demostrando que aún no ha iniciado la transición, continúa expresándose con un un ``tono agudo, infantil.'' La dualidad infante/adulto de este fragmento se resignifica en el marco cultural del autor.

 %FIXME: referencia, no, nada introductorio, relacionar con tópico
Asímismo, fue este viaje según el cuál, la estoica y físicamente virtuosa apariencia de la estatua \emph{El auriga de Delfos}, lo llevo a reestructurar su vida alrededor del paradigma del cuerpo, en contraposición a la vida de la mente que había llevado hasta entonces a costa de la humillación producida por su débil aspecto. Aunado al hecho de en aquel contexto Japón experimentaba un agudo choque cultural con occidente debido a la ocupación norteamericana, el autor, mediante su obra \emph{El rumor del oleaje} propone un regreso al idílico pasado de Japón en el cual el \emph{modus vivendi} de las personas debe procurar una profunda integración con la naturaleza para demostrar virtud, tal como hace el personaje protagónico \emph{Shinji} y en menor medida el resto de los habitantes de \emph{Uta Jima}. %explicar

 Respecto a esta integración ambiente-personaje inmediatamente resalta la manera en que el autor configura la trama de la novela alrededor de eventos cuyo orígen reside exclusivamente en la naturaleza. De este modo, los sucesos narrados obedecen no solo a la lógica derivada de la manera en que los personajes se relacionan entre sí, sino que también del tipo de relación que mantienen con la naturaleza.

  La localidad en que transcurren los hechos narrados, \emph{Uta Jima}, se organiza  de acuerdo a este paradigma, mediante la deificación panteísta de la naturaleza --como un reflejo del sintoísmo japonés--. %FIXME:cita
