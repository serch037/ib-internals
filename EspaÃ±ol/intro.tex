% Estímulo ¿De qué recursos se vale el autor para desarrollar los temas?
% Título: La topología psicológica en \emph{El rumor del oleaje} de Yukio Mishima
Yukio Mishima es considerado uno de los más importantes escritores japoneses.
%Su vida se marcó por múltiples contradicciones, así como por un intento de encontrar un punto de equilibrio entre los puntos opuestos que lo marcaban.%FIXME ¿referencia? no
Su novela, \emph{El rumor del oleaje}, publicada en 1954, nos permite percibir cómo Mishima se imaginaba un estilo de vida íntegro y en equilibrio ideal con la naturaleza, mediante la idealización de la masculinidad representada por la formación como adulto del personaje protagónico Shinj; proceso condensado en la escena de la torre de observación.

El narrador representa a Shinji como un joven virtuoso en cuerpo y en espíritu, respetuoso y moderado en sus ambiciones, pero dispuesto a atravesar cualquier obstáculo para conseguirlas. Sus dotes físicos, tienen origen en su pesado trabajo como pescador:``era alto y fornido para su edad, únicamente sus facciones revelaban su juventud [\ldots] Sus ojos azules eran muy claros,[\ldots] ese don que el mar concede a quienes se ganan en él su sustento''\autocite{mishima2006}; una labor que implica una estrecha relación con la naturaleza.

La novela, que es una libre adaptación del clásico griego \emph{Dafnis y Cloe}, se estructura a modo de \emph{Bildungsroman}, o novela de formación.
Por lo tanto, entre los motivos de la novela destaca la educación sexual de los personajes principales --Shinju y Hatsue-- que marca la transición desde la juventud a la adultez de los mismos.
En su primer encuentro, podemos apreciar la virtud de la inocencia en Shinji, quien en un momento casi ridículo: ``pasó a propósito por delante de ella, y de la misma manera en que los niños se quedan mirando un objeto extraño, se detuvo y la miró a la cara.''\autocite{mishima2006}
La experiencia le trastorna por completo, pero su candor juvenil le impide percibir que se ha enamorado: ``yació preguntándose qué le ocurría, temeroso de que aquello [\ldots] [fuera] estar enfermo.''\autocite{mishima2006}

Por el enfoque narrativo, centrado en Shinji, podemos percatarnos que es este personaje quien realiza de manera activa la transformación a la adultez, mientras que Hatsue permanece como un personaje secundario cuya presencia es complementaria al proceso de transición del primero y cuyo crecimiento es pasivo en comparación.
Para construir este proceso --y que el héroe concluya su metamorfosis en el ideal de lo masculino--, el narrador le presenta numerosas pruebas físicas y morales, cada una de las cuales debe superar para demostrar su valor. Entre estas pruebas, destaca el momento en que se reúnen en la torre de observación. 

%Es en esta escena en la que aparentemente por accidente, yace en el mismo lecho que Hatsue, y a petición de esta, así como por su falta de experiencia que lo mantiene ignorante respecto las cuestiones morales relacionadas con las mujeres: ``sentía un respeto aleatorio hacia las actitudes morales. Y como aún no había conocido íntimamente a una mujer, creyó haber llegado a lo más profundo del ser de Hatsue, donde radicaba su moralidad, y no insistió más.'' limita sus avances --cuyos detalles el narrador ofusca de manera deliberada--  para no contrariarla.

Esta escena, aunque breve, resulta particularmente interesante por el contenido simbólico que el narrador establece a partir de los elementos de la naturaleza, mediante los cuales ilustra la falta de preparación de los jóvenes para la consumación de su amor.
La manera en que el narrador relaciona la naturaleza con los personajes funciona como un indicio del alto nivel de integración que estos mantienen con su entorno. 
Así, la escena se desarrolla rodeados de una sensación de completa naturalidad.

El fragmento comienza con Shinji despertándose tras una breve siesta en la torre de observación, factor que le brinda a la escena cualidades oníricas, de hecho, el propio personaje: ``se preguntó si estaba soñando''.\autocite{mishima2006}
Cabe mencionar que el narrador presenta la torre como el refugio idóneo de una intensa tormenta que está teniendo lugar, misma que presagia la relevancia y viveza de los acontecimientos que tendrán lugar.

Lo primero de lo que se percata es del fuego frente a sí, que le revela la sombra de una figura desvistiéndose. El fuego, que simboliza el deseo, permea el ambiente y le guía hacia Hatsue al mismo tiempo que le brinda a esta un pretexto para desvestirse: ``la idea de que se estaba desvistiendo delante de un hombre no había pasado por su mente. Sencillamente, se desvestía ante una fogata porque era el único fuego disponible y porque estaba mojada.'' \autocite{mishima2006}
Shinji se aprovecha de esta situación y decide observarla por el rabillo del ojo, sin embargo, Hatsue se percata de esto por la proyección de la sombra de las pestañas de Shnji.
No es sorprendente que sea el mismo fuego que antes le presentó la figura de Hatsue aquel que lo delate, ya que, en otra manifestación de su relación simbólica con el erotismo, refleja el despertar del deseo en Hatsue.
Aquí, el narrador aprovecha para recordarnos que a pesar de haber actuado como un \emph{voyeurista}, la virtud de Shinji no ha sido perturbada: ``el honesto muchacho cerró los ojos y los apretó con fuerza. Ahora que lo pensaba, desde luego había sido un error [observar a Hatsue] [\ldots]''\autocite{mishima2006}.
Sin embargo, tras una breve examinación de conciencia, decide abrir los ojos en un acto que resulta simultáneamente desafiante y desconcertante.
Simbólicamente, ese ``abrir de ojos'', lo encamina en su transformación como figura adulta, una libre del pudor infantil ``no comprendía que, por el mero hecho de que estuviese desnuda, se había alzado una barrera entre ellos que dificultaba las muestras de cortesía ordinarias, las confianzas naturales''.\autocite{mishima2006}
En cambio, Hatsue demuestra que aún no ha iniciado la transición, ya que continúa expresándose con un ``tono agudo, infantil.''\autocite{mishima2006}

A pesar de antes haber fungido como cómplice, el deseo sensual --simbolizado por el fuego-- rápidamente muestra sus limitaciones como conductor de crecimiento entre los inmaduros jóvenes: ``se quedaron quietos, mirándose, separados por las llamas [\ldots] alzándose para siempre entre ellos.''.\autocite{mishima2006}
El narrador acentúa por medio de esta hipérbole el hecho de que a pesar de verse inmersos en anhelos, estos no pueden ser aún satisfechos. En consecuencia, contrario a las señales de madurez que había aparentado, Shinji demuestra de nuevo sus cualidades infantiles al iniciar un juego con Hatsue que les lleva a desvestirse.

A pesar de esta iniciativa, Shinji, por un comentario de Hatsue, recupera momentáneamente el pudor, pero la tensión erótica, lo impulsará a un último atrevimiento: ``[se acercó] tanto al fuego que casi se le quemaron las yemas de los dedos, y mirando con fijeza la camisa de la muchacha, en la que oscilaban las sombras arrojadas por las llamas, por fin logró decirle:
—Si [\ldots] si apartas eso [refiriéndose al camisón de la muchacha]\ldots lo haré. [desnudarse]''\autocite{mishima2006}. Así, el fuego o deseo, alcanza tales magnitudes que se vuelve casi doloroso, pero le infunde el valor suficiente a Shinji para exigirle a la joven que se desnude por completo.
El resultado de este juego dejaf perplejos a ambos: ``En los labios de Hatsue afloró una sonrisa espontánea, pero ni ella ni Shinji tenían la menor idea de cuál podría ser el significado de su sonrisa''\autocite{mishima2006}, así, ocurre un desdoblamiento en Hatuse entre su ego infantil e inocente y su yo adulto. 

De manera sinérgica a la intensificación de la atracción entre Shinji y Hatuse la tormenta igualmente se acriecenta. Este momento representa un punto de quiebre en el cuál los enamorados, por su aún latente inmadurez, comienzan a perder el control de su situación: ``La muchacha retrocedió unos pasos [\ldots] No había salida.''\autocite{mishima2006}. Para demostrar las señales de su crecimiento, el narrador le presenta una prueba adicional a Shinji, desarmar el deseo inspirado por Hatsue, que simultáneamente le inquieta y le paraliza; entonces ella, en un estremecimiento de la tormenta que insinúa la ceguera por el deseo, le ordena al muchacho: ``Salta por encima del fuego y ven aquí. ¡Vamos! Si saltas por encima del fuego y vienes\ldots''\autocite{mishima2006}. Shinji obedece y ágilmente sortea las llamas, después de lo cual cae sobre ella, pero para evitar su prematura unión, interrumpe la naturaleza ``Se abrazaron. [\ldots] —Las agujas de pino [\ldots] —dijo Hatsue—. Hacen daño. [\ldots] —Es malo. ¡Es malo!''\autocite{mishima2006} Naturalmente, Shinji acepta cuando Hatsue le dice que no deben continuar. Sin embargo el narrador aquí intercede por Shinji para recordarnos que no fue mediante el hecho de haber obedecido a una casi arbitraria regla social: ``le preguntó sin convicción el alicaído muchacho''\autocite{mishima2006} que Shinji demostró su virtud, sino que por el hecho de haber respetado la moralidad de los Hatuse en un acto que provino de un profundo respeto hacia ella. 

Al concluir esta escena de gran tensión dramática, el ambiente y los personajes se calman, en una manera igualmente sinérgica a su anterior arrobamiento. En primer lugar, disminuye el ardor del fuego: ``[\ldots] el fuego moribundo crepitaba un poco.''\autocite{mishima2006}; en seguida se hace evidente la profunda integración de los personajes con la naturaleza, indicio de que han superado exitosamente el desafío y el peligro de corromperse ha pasado: ``Oían ese sonido y los silbidos del viento al pasar [\ldots] todo ello mezclado con los latidos de sus corazones.''\autocite{mishima2006}

En conclusión, la escena de la torre de observación es relevante dentro de la configuración de la novela como un \emph{Bildungsroman}, ya que no solo conforma un importante capítulo en el proceso de metamorfosis desde la juventud hacia la adultez de Shinji, sino que concentra la esencia del mismo: una serie de pruebas que este personaje debe superar para demostrar su virtud. Asímismo, esta escena sugiere por la relación simbólica entablada  entre la narración dramática y los elementos de la naturaleza uno de los más importantes valores que nos permite conocer la novela: la integridad de los individuos con la naturaleza. 
