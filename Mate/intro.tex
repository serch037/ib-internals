\Large{\textbf{Análisis de la complejidad computacional del algoritmo Minimax}}

\normalsize
El juego es parte fundamental de las sociedades y las culturas del hombre.
Naturalmente existen de diversos tipos, pero lo que todos tienen en común es el
hecho de que estimulan el pensamiento estratégico de  los jugadores. Por lo
tanto, la investigación sobre cómo maximizar la probabilidad de victoria en
ellos ha sido objeto de estudio por siglos. No en balde la rama de probabilidad
inició como una descripción de los juegos de azar. 

Por esta misma razón, desde el surguimiento de las computadoras, uno de los
principales objetivos de los computólogos ha sido la creación de inteligencias
artificiales programadas para resolver este tipo de problemas. Por esta razón
comenzó a interesarme el tema del algoritmo Minimax, ya que me parece un
procedimiento cuyo planteamiento es simultáneamente simple y poderoso. El área
de inteligencia artificial, que combina de manera interdisciplinaria
concocimientos de distintas ramas, como matemáticas e informática, me parece
sumamente interesante, y siempre ha llamado mi atención. Recuerdo la fascinación
que produjo en mí aprender sobre el enfrentamiento entre la computadora Deep
Blue y el maestro Garry Kasparov, durante el cuál se desafió el escepticismo de
muchos ante las posibilidades que podrían brindar las computadoras analizando
situaciones complejas como el juego de ajedrez. 

Sin embargo los juegos no son más que una planta de desarrollo, ya que los
principios detrás de los algoritmos utilizados, por ejemplo, la rama
matemática detrás de la estrategia que muchos utilizan, teoría de juegos, fue responsable por algunas
decisiones importantes durante la guerra fría, y resulta una herramienta
efectiva para la toma de decisiones bajo diversas circunstancias. 

De esta manera, decidí indagar sobre las técnicas fundamentales detrás del éxito
revolucionario de Deep Blue, esta investigación me llevó a aprender, entre otras
cosas, sobre la
estrategia Minimax y sobre la teoría detrás de esta misma. Y debido a  mi profundo interés en
la programación me percaté de que implementar un algoritmo similar seria un
excelente reto para aprender más sobre los fundamentos de una de las ramas que
me parecen más interensantes del desarrollo tecnológico contemporáneo. 

Lo que me propuse en esta investigación fue aprender sobre el algoritmo Minimax
e implementarlo en un pequeño programa, asímismo, me interesó conocer
sobre su comportamiento en distintos escenarios, particularmente sobre los
requerimientos computacionales de tiempo y espacio para hallar la solución
óptima bajo distintos escenarios para determinar posibles limitaciones del procedimiento. 

