\documentclass[12pt,letterpaper,oneside,DIV=13]{scrartcl}
\usepackage{paralist}
\usepackage{ifpdf}
\usepackage{url}

%Language Settings
\usepackage[
    spanish, mexico]
    %english]
    {babel}

    \usepackage[
spanish=mexican,
    %american,
    autostyle=true
    ]{csquotes}


\usepackage[utf8]{inputenc}
\usepackage{fontspec}
\usepackage{setspace}
\setsansfont[Ligatures=TeX]{Lato}
%\usepackage{sectsty}
\setmainfont[Ligatures=TeX]{Minion Pro}
\usepackage{microtype}
\usepackage{hyperref}
\usepackage{enumitem}
\usepackage[document]{ragged2e}
\usepackage{amsmath}
\usepackage{tikz}


\usepackage{booktabs}
\usepackage{epigraph}
%\usepackage{kpfonts}
\usepackage[notes,
%autocite=footcite,
url=false
]{biblatex-chicago}
%\addbibresource{biblio.bib}
%\addbibresource{extra.bib}

%Fix parenthesis

\title{Curvas elípticas y criptografía}
\subtitle{Exploración Matemática}
\author{Sergio Ugalde Marcano}


\usepackage{paralist}
\begin{document}
\maketitle
\thispagestyle{empty} 
\begin{flushleft}
\onehalfspacing
\setlength{\parindent}{0.5in} 
\begin{center}
	\begin{tikzpicture}[domain=-1.769292354238631:2.5, samples at ={-1.769292354238631, -1.76, -1.74, ..., 2.26, 2.35, 2.7, 2.9}]
	%	\draw[very thin,color=gray] (-2,-4) grid (3,4);
        \draw[very thin,color=gray] (-1.9,-3.9) grid (3.9,3.9);

		\draw[->] (-2.2,0) -- (3.2,0) node[right] {$x$};
		\draw[->] (0,-2.2) -- (0,4.2) node[above] {$y$};
		\draw[->, color=red] plot (\x,{sqrt(\x^3-2*\x+2)}) node[right] {$y^2=x^3-2x+2$};
		\draw[->, color=red] plot (\x,{-sqrt(\x^3-2*\x+2)}) node[right] {};
		
%		\draw[<->, color=blue] (-2.1, 1.75745899181) -- (3.1, 1.75745899181) node[above] {};
%		\draw[color=blue] (-.816496580928, 1.75745899181) node[above] {$p$};
%		\draw[color=blue] (-.816496580928, 1.75745899181) node[] {$\bullet$};
%		\draw[color=blue] (1.63299316186,  1.75745899181) node[above] {$p^\star$};
%		\draw[color=blue] (1.63299316186,  1.75745899181) node[] {$\bullet$};
	\end{tikzpicture}
\end{center}
\onehalfspacing
\frenchspacing
%\nocite{*}
\end{flushleft}
%\input{citations}
\end{document}
